\documentclass[aspectratio=169]{beamer}
\usepackage{color,amsmath,amssymb,graphicx,subcaption,geometry,mathtools,xfrac}
%\usepackage{cite}
\usepackage{mhchem}
\usepackage{tikz}
\usepackage{pgfplots}
\usepackage{lineno}
\pgfplotsset{compat=1.12}
\usepackage{stackengine,ifthen}
\usepackage{float}
\usetikzlibrary{arrows,positioning,calc,arrows.meta,patterns,fit}

\newtoggle{article}
\newtoggle{poster}
\newtoggle{eddpathway}
\newtoggle{elifesubmission}
\newtoggle{thesis}
\togglefalse{elifesubmission}
%\toggletrue{elifesubmission}

\iftoggle{elifesubmission} {
\usepackage{setspace}
\linenumbers
\doublespacing
\newcommand{\beginsupplement}{%
        \setcounter{table}{0}
        \renewcommand{\thetable}{S\arabic{table}}%
        \setcounter{figure}{0}
        \renewcommand{\thefigure}{2-figure supplement \arabic{figure}}%
    }
}
{
\newcommand{\beginsupplement}{%
        \setcounter{table}{0}
        \renewcommand{\thetable}{S\arabic{table}}%
        \setcounter{figure}{0}
        \renewcommand{\thefigure}{S\arabic{figure}}%
    }
}


\tikzset{>=Latex}
\newcommand\influx{0.5}

\newenvironment{customlegend}[1][]{
  \begingroup
  \csname pgfplots@init@cleared@structures\endcsname
  \pgfplotsset{#1}
}{
  \csname pgfplots@createlegend\endcsname
  \endgroup
}

\def\addlegendimage{\csname pgfplots@addlegendimage\endcsname}
\setlength\abovecaptionskip{6pt}
\providecommand{\abs}[1]{\lvert#1\rvert}
\providecommand{\norm}[1]{\lVert#1\rVert}

\tikzset{>=latex}
\tikzset{metaboliteStyle/.style={}}
\definecolor{cyan}{RGB}{100,181,205}
\definecolor{blue}{RGB}{76,114,176}
\definecolor{green}{RGB}{85,168,104}
\definecolor{magenta}{RGB}{129,114,178}
\definecolor{yellow}{RGB}{204,185,116}
\definecolor{red}{RGB}{196,78,82}
\definecolor{graybg}{gray}{0.95}

\colorlet{assimcol}{green}
\colorlet{sumcolor}{yellow}

\colorlet{inputcol}{green}
\colorlet{branchout}{red}
\colorlet{branchoutfl}{red!80}
\colorlet{autocatacyc}{blue}
\colorlet{autocatacycfl}{blue!80}
\colorlet{autocataby}{cyan}

\pdfpageattr{/Group <</S /Transparency /I true /CS /DeviceRGB>>} 

\def\blendfrac{0.5}
\def\deltaang{-155}
\def\fromang{180}
\def\inputang{-40}
\def\protrude{7}
\def\arcwidth{0.3cm}
\def\highlightrad{0.2cm}
\def\autocatalrad{1.5cm}
\def\autocatalscale{1.5}

  \newcommand{\colorgradarc}[6]{%width,startang,stopang,rad,startcol,stopcol
    \pgfmathsetmacro\arcrange{#3-#2}
    \pgfmathsetmacro\progsign{\arcrange>0 ? 1 : -1}
    \pgfmathsetmacro\arcend{#3-1}
    \foreach \i in {#2,...,\arcend} {
     \pgfmathsetmacro\fracprog{\i/\arcrange-#2/\arcrange}
     \pgfmathsetmacro\col{\fracprog*100}
     \draw[color={#6!\col!#5},line width=#1] (\i-\progsign:#4)
         arc[start angle=\i-\progsign, end angle=\i+1.1*\progsign,radius=#4];
    }
  }

  \newcommand{\shadedarc}[7][\arcwidth]{%width,startang,stopang,startrad,stoprad,startcol,stopcol
    \pgfmathsetmacro\arcrange{#3-#2}
    \pgfmathsetmacro\radrange{#5-#4}
    \pgfmathsetmacro\progsign{\arcrange>0 ? 1 : -1}
    \foreach \i in {#2,...,\numexpr#3-1\relax} {
      \pgfmathsetmacro\fracprog{\i/\arcrange-#2/\arcrange}
      \pgfmathsetmacro\col{\fracprog*100}
      \draw[color={#6!\col!#7},line width=#1] (\i-\progsign:#4+\radrange*\fracprog)
  arc[start angle=\i-\progsign, end angle=\i+1.1*\progsign,radius=#4+\fracprog*\radrange];
    }
  }

  \newcommand{\preassim}[5]{%width, startang, stopang, rad, col
    \pgfmathsetmacro\halfarc{#3/2-#2/2}
    \draw[color={autocatacyc},line width=#1] (#3:#4)
        arc[start angle=#3, end angle=#3-\halfarc,radius=#4];
    \pgfmathsetmacro\startshade{#2}
    \pgfmathsetmacro\endshade{#3-\halfarc}
    \colorgradarc{#1}{\startshade}{\endshade}{#4}{#5}{autocatacyc}
  }

  \newcommand{\postassim}[6]{%width,startang,stopang,rad,col,ratio
    \pgfmathsetmacro\halfarc{#3/2-#2/2}
    \pgfmathsetmacro\progsign{\halfarc>0 ? 1 : -1}
    \pgfmathsetmacro\quarterarc{#3/4-#2/4}
    \pgfmathsetmacro\endshade{#3-\halfarc}
    \pgfmathsetmacro\startbranch{#3-\quarterarc}
    \colorgradarc{#1*#6}{#2}{\endshade}{#4+#1*#6/2-#1/2}{autocatacyc}{#5}
    \draw[color={#5},line width=#1*#6] (\endshade:#4+#1*#6/2-#1/2)
        arc[start angle=\endshade, end angle=\startbranch+\progsign,radius=#4+#1*#6/2-#1/2];
    \draw[color={#5},line width=#1] (\startbranch:#4)
        arc[start angle=\startbranch, end angle=#3,radius=#4];

    \pgfmathsetmacro\arcrange{-\quarterarc}
    \pgfmathsetmacro\outstart{\startbranch+180}
    \pgfmathsetmacro\progsign{\arcrange>0 ? 1 : -1}
    \pgfmathsetmacro\arcend{\outstart+\arcrange-\progsign}
    \pgfmathsetlength\arrowwidth{#1*#6-#1}
    \begin{scope}[shift={(\startbranch:2*#4+#1*#6/2)}]
      \draw[color={#5},line width=#1*#6-#1] (\outstart:#4)
          arc[start angle=\outstart, end angle=\arcend,radius=#4];
      \revarrowhead{\arrowwidth}{\arcend}{#4}{#5}
    \end{scope}
  }

  \newcommand{\assim}[5]{%width,startang,deltaang,rad,ratio
    \pgfmathsetmacro\assimstart{#2+180}
    \begin{scope}[shift={(#2:2*#4+#1*#5/2)}]
      \colorgradarc{#1*#5-#1}{\assimstart}{\assimstart+#3}{#4}{autocatacyc}{assimcol}
    \end{scope}
  }

  \newcommand{\arrowhead}[4]{%width,startang,rad,col
  \fill[#4] (#2+1:#3-#1/2) arc (#2+1:#2:#3-#1/2)
       -- (#2-\protrude:#3) -- (#2:#3+#1/2) arc (#2:#2+1:#3+#1/2) -- cycle;
     }
  \newcommand{\revarrowhead}[4]{%width,startang,rad,col
  \fill[#4] (#2-1:#3-#1/2) arc (#2-1:#2:#3-#1/2)
       -- (#2+\protrude:#3) -- (#2:#3+#1/2) arc (#2:#2-1:#3+#1/2) -- cycle;
     }



\renewcommand{\footnoterule}{}
\tikzset{
  invisible/.style={opacity=0},
  visible on/.style={alt={#1{}{ invisible}}},
  alt/.code args={<#1>#2#3}{%
    \alt<#1>{\pgfkeysalso{#2}}{\pgfkeysalso{#3}} 
  },
}


\usepackage{adjustbox}
\togglefalse{article}
\togglefalse{eddpathway}
\togglefalse{elifesubmission}
\setbeamertemplate{footline}[frame number]

\newcommand{\backupbegin}{
  \newcounter{finalframe}
    \setcounter{finalframe}{\value{framenumber}}
}

\newcommand{\backupend}{
  \setcounter{framenumber}{\value{finalframe}}
}

\title{The interplay between metabolic network topology and the kinetic parameters of enzymes, from autocatalytic cycles and beyond}
\author{Uri Barenholz}
\institute{Department of Plant \& Environmental Sciences\\
Weizmann Institute of Science, Rehovot, Israel}
\date{November 3, 2017}
\usepackage[absolute,overlay]{textpos}
\newcommand\urltext{
    \begin{textblock*}{\paperwidth}(0pt,\textheight)
        \raggedright \small \url{https://git.io/vFO0r} \hspace{.5em}
    \end{textblock*}
}
\begin{document}

\newlength\gridsize
\pgfmathsetlength{\gridsize}{8cm}
\newlength\plotwidth
\pgfmathsetlength{\plotwidth}{4.6cm}
\newlength\plotheight
\pgfmathsetlength{\plotheight}{4.5cm}
\newlength\plotwidthanim
\pgfmathsetlength{\plotwidthanim}{8cm}
\newlength\plotheightanim
\pgfmathsetlength{\plotheightanim}{7cm}
\newlength\plotshift
\pgfmathsetlength{\plotshift}{-6cm}
\newlength\nodedist
\pgfmathsetlength{\nodedist}{2.7cm}
\newcommand{\fontsizedef}{\Large}
\newcommand{\ratiosizedef}{\huge}

\newlength\assimwidth

\newlength\cbbimrad
\newlength\cbbierad
\newlength\cbbesrad
\newlength\cbbemrad
\newlength\cbbeerad
\newlength\cbbwidth
\newlength\cbbtotwidth

\newlength\glyimrad
\newlength\glyierad
\newlength\glyesrad
\newlength\glyemrad
\newlength\glyeerad
\newlength\glywidth
\newlength\glytotwidth
\newlength\glyfinwidth
\newlength\glyimmrad
\newlength\glyemmrad
\newlength\glyemmmrad
\newlength\glyeamrad

\newlength\ptsierad
\newlength\ptsimrad
\newlength\ptsarcwidth
\newlength\ptsesrad
\newlength\ptsemrad


\newlength\headlinedist
\newlength\arrowwidth


\newlength\midwidth
\newlength\midrad
\pgfmathsetlength{\midwidth}{\arcwidth*1.5}
\pgfmathsetlength{\midrad}{\autocatalrad+\arcwidth/4}


\frame{
  \titlepage
    \urltext
}

\frame{\frametitle{An autocatalytic cycle requires its internal metabolite to produce it}
    \begin{adjustbox}{max totalsize={\textwidth}{\textheight},center}
        \begin{tikzpicture}
  \colorlet{genext}{assimcol}
  \colorlet{genmed}{blue}
  \colorlet{geninit}{blue}

  \newlength\imrad;
  \newlength\ierad;
  \newlength\esrad;
  \newlength\emrad;
  \newlength\eerad;
  \pgfmathsetlength{\imrad}{\autocatalrad-\blendfrac*\arcwidth};
  \pgfmathsetlength{\ierad}{\autocatalrad-0.5*\arcwidth};
  \pgfmathsetlength{\esrad}{\autocatalrad+\arcwidth};
  \pgfmathsetlength{\emrad}{\autocatalrad+\arcwidth-\blendfrac*\arcwidth};
  \pgfmathsetlength{\eerad}{\autocatalrad+0.5*\arcwidth};

  \preassim{\autocatalscale*\arcwidth}{-100}{-270}{\autocatalscale*\autocatalrad}{geninit};%width, startang, stopang, rad, col
    \postassim{\autocatalscale*\arcwidth}{90}{-45}{\autocatalscale*\autocatalrad}{geninit}{2}
    \assim{\autocatalscale*\arcwidth}{90}{-30}{\autocatalscale*\autocatalrad}{2}
    \arrowhead{\autocatalscale*\arcwidth}{-45}{\autocatalscale*\autocatalrad}{geninit}

    \node[align=center] at (-30:\autocatalscale*\autocatalrad*2.1) (int) {$\delta \cdot$Internal metabolite};
    \node[align=center] at (-73:\autocatalscale*\autocatalrad*1.05) (int) {Internal\\metabolite};
    \node[align=center] at (130:\autocatalscale*\autocatalrad*1.65) (ext) {Assimilated\\metabolite};
    \iftoggle{poster}{
    \node [above left=2mm and -18mm of ext,align=center] (eq1) {Internal\\metabolite};
    \node [right=1mm of eq1] (eq2) {+};
    \node [right=1mm of eq2,align=center] (eq3) {Assimilated\\metabolite};
    \node [right=1mm of eq3] (eq4) {$\rightarrow (1+\delta)$};
    \node [right=1mm of eq4,align=center] (eq5) {Internal\\ metabolite};
        \node [rectangle,fill=autocatacyc!30,rounded corners=3pt,fit=(eq1) (eq5)] {};
    \node [above left=2mm and -18mm of ext,align=center] (eq1) {Internal\\metabolite};
    \node [right=1mm of eq1] (eq2) {+};
    \node [right=1mm of eq2,align=center] (eq3) {Assimilated\\metabolite};
    \node [right=1mm of eq3] (eq4) {$\rightarrow (1+\delta)$};
    \node [right=1mm of eq4,align=center] (eq5) {Internal\\ metabolite};
    }{
    \node [rectangle,fill=autocatacyc!30,rounded corners=3pt] at (90:\autocatalscale*\autocatalrad+1.7cm) (eq) {Internal metabolite + Assimilated metabolite $\rightarrow (1+\delta)$ Internal metabolite};
}

\end{tikzpicture}


    \end{adjustbox}
    %presentation link on github
    % https://git.io/vSodL
    \urltext
}

\frame{\frametitle{Why do we care about autocatalytic cycles?}
\begin{itemize}
    \item The lab implements the Calvin-Benson-Bassham cycle in \emph{E.coli}\footnote{Antonovsky et. al., Cell 2016}
    \item Two enzymes were introduced
    \item It didn't work
    \item Can we understand why?
\end{itemize}
}

\frame{\frametitle{Stable flux through an autocatalytic cycle constrains the kinetic parameters of its enzymes}
    \begin{adjustbox}{max totalsize={\textwidth}{0.8\textheight},center}
          \begin{tikzpicture}[>=latex',node distance = 2cm]
    \tikzset{
        vstyle/.style={opacity=0.3,pattern=north west lines,cyan,visible on=<7->}}
    \tikzset{
        kstyle/.style={opacity=0.3,pattern=north east lines,magenta,visible on=<7->}}
  \begin{scope}[shift={(-4cm,4.3cm)}]
        \node at (-60:1cm) (X) {$X$};
        \node[shape=coordinate] (orig) {};
        \draw [-,line width=1pt,autocatacyc] (X.south west) arc (285:0:1cm) node [pos=0.65,above] (fa) {$f_a:$\small{$A+X\rightarrow2X$}} node [pos=0.45,shape=coordinate] (midauto) {} node [pos=1,shape=coordinate] (endcommon) {};
        \draw [->,line width=1pt,autocatacyc] (endcommon) arc (-25:-44:2cm);
        \draw [->,line width=1pt,autocatacyc] (endcommon) arc (-5:-32:1.5cm);
        \draw [line width=1pt,assimcol] (midauto) arc (-60:-90:1cm) node [pos=1,left] (e) {$A$};
        \draw [->,line width=1pt,branchout] (X.south east) arc (225:270:1cm) node [pos=0.75,above] {$f_b$};
        \iftoggle{article} {
            \node at (-2.4cm,1.3cm) (A) {(A)};
        }{}
  \end{scope}
  \begin{scope}[shift={(-1.5cm,-\gridsize/2)}]
    \begin{axis}[name=phase,clip=false,xmin=0,ymin=0,xmax=2,ymax=2,ylabel={\Large{$\sfrac{V_{\max,b}}{V_{\max,a}}$}},xlabel={\Large{$\sfrac{K_{M,b}}{K_{M,a}}$}},samples=6,width=\gridsize,height=\gridsize,ytick={0,1,2},xtick={0,1,2},visible on=<7->]
        \addplot[domain=0:2,dotted,black,thick] {x};
        \addplot[dotted,black,thick] coordinates {(0,1) (2,1)};
        \draw[kstyle] (axis cs:0,0) -- (axis cs:2,2) -- (axis cs:2,0) --cycle;
        \draw[vstyle] (axis cs:0,1) -- (axis cs:2,1) -- (axis cs:2,2) -- (axis cs:0,2) --cycle;
        \draw[->,black!50,dashed] (axis cs:0.25,1.4) -- +(-1.9cm,0cm);
        \draw[->,black!50,dashed] (axis cs:1.75,1.4) -- +(1.1cm,0cm);
        \draw[->,black!50,dashed] (axis cs:0.25,0.6) -- +(-1.9cm,0cm);
        \draw[->,black!50,dashed] (axis cs:1.75,0.6) -- +(1.1cm,0cm);
        \node[align=left,anchor=east] at (axis cs:1.75,1.4) (I) {I};
        \node[align=right,anchor=west] at (axis cs:0.25,1.4) (II) {II};
        \node[align=right,anchor=west] at (axis cs:0.25,0.6) (III) {III};
        \node[align=left,anchor=east] at (axis cs:1.75,0.6) (IV) {IV};
      \end{axis}

\iftoggle{article} {
        \pgfmathsetlength{\plotwidthanim}{\plotwidth}
        \pgfmathsetlength{\plotheightanim}{\plotheight}
        \pgfmathsetlength{\plotshift}{1mm}
}{
    \only<5-> {
        \pgfmathsetlength{\plotwidthanim}{\plotwidth}
        \pgfmathsetlength{\plotheightanim}{\plotheight}
        \pgfmathsetlength{\plotshift}{1mm}
    }
}

      \begin{axis}[name=plot1,axis x line=middle,axis y line=left,xlabel near ticks,ylabel near ticks,xmin=0,ymin=-2.5,xmax=2.9,ymax=5.9,xlabel={[$X$]},ylabel={flux},samples=60,width=\plotwidthanim,height=\plotheightanim,clip=false,yticklabels={,,},xticklabels={,,},tick label style={major tick length=0pt},at=(phase.right of north east),anchor=left of north west,ylabel style={name=ylabel1},xshift=\plotshift,visible on=<2->]%,axis background/.style={fill=cyan!50!magenta,opacity=0.3}]
        \addplot[domain=0:2.9,autocatacyc,thick] {3*x/(0.1+x)};
        \addplot[domain=0:2.9,branchout,thick,visible on=<3->] {5*x/(1+x)};
        \addplot[domain=0:2.9,sumcolor,thick,visible on=<4->] {3*x/(0.1+x)-5*x/(1+x)};
        \addplot[dashed,gray,thick,visible on=<4->] coordinates {(1.25,0) (1.25,2.77)};
        \node[right,align=left,visible on=<6->] (onetext) at (axis cs:0.05,4.7) {\scriptsize \textbf{stable non-zero}\\[-0.4em]\scriptsize \textbf{steady state}};
      \end{axis}
     \iftoggle{elifesubmission} {}
     {
      \iftoggle{article} {}
      {
        \node[visible on=<2-4>,color=blue,at=(plot1.left of north west),anchor=north east,scale=1.5,xshift=-1cm,yshift=-0.5cm] (fa){$f_a=\frac{V_{\max,a}X}{K_{M,a}+X}$};
        \node[visible on=<3-4>,color=red,below=of fa,scale=1.5,yshift=0.7cm] (fb) {$f_b=\frac{V_{\max,b}X}{K_{M,b}+X}$};
      }
    }
      \node[draw,fit=(plot1) (ylabel1),line width=2pt, fill=none,rounded corners=3pt,cyan!50!magenta,opacity=0.6,visible on=<7->]{};

        \begin{customlegend}[legend entries={$f_a$,$V_{\max,b}>V_{\max,a}$,$f_b$,$\sfrac{V_{\max,b}}{V_{\max,a}}<\sfrac{K_{M,b}}{K_{M,a}}$,$\dot{X}=f_a-f_b$},legend style={above=1cm of plot1.north east,anchor=south east,name=legend1,visible on=<4->},legend columns=2]
          \addlegendimage{autocatacyc,fill=black!50!red,sharp plot,line width=1pt}
          \addlegendimage{vstyle,area legend,visible on=<5->}
          \addlegendimage{branchout,fill=black!50!red,sharp plot,line width=1pt}
          \addlegendimage{kstyle,area legend,visible on=<5->}
          \addlegendimage{sumcolor,fill=black!50!red,sharp plot,line width=1pt}
        \end{customlegend}

      \begin{axis}[name=plot2,axis x line=middle,axis y line=left,xlabel near ticks,ylabel near ticks,xmin=0,ymin=-2.5,xmax=2.9,ymax=5.9,xlabel={[$X$]},ylabel={flux},samples=60,at=(phase.left of north west),anchor=right of north east,width=\plotwidth,height=\plotheight,yticklabels={,,},xticklabels={,,},tick label style={major tick length=0pt},ylabel style={name=ylabel2},xshift=-1mm,visible on=<5->]%,axis background/.style=vstyle]
        \addplot[domain=0:4,autocatacyc,thick] {4*x/(1+x)};
        \addplot[domain=0:4,branchout,thick] {5*x/(0.2+x)};
        \addplot[domain=0:4,sumcolor,thick,visible on=<6->] {4*x/(1+x)-5*x/(0.2+x)};
        \node[right,align=left,visible on=<6->] (twotext) at (axis cs:0.0,5) {\scriptsize stable zero steady state};
      \end{axis}
     \iftoggle{article} {}
     {
      \node[draw,fit=(plot2) (ylabel2),line width=2pt, vstyle,fill=none,rounded corners=3pt]{};
  }

      \begin{axis}[name=plot3,axis x line=middle,axis y line=left,xlabel near ticks,ylabel near ticks,xmin=0,ymin=-2.5,xmax=2.9,ymax=5.9,xlabel={[$X$]},ylabel={flux},samples=60,width=\plotwidth,height=\plotheight,yticklabels={,,},xticklabels={,,},tick label style={major tick length=0pt},at=(phase.left of south west),anchor=right of south east,ylabel style={name=ylabel3},xshift=-1mm,visible on=<5->]
        \addplot[domain=0:4,autocatacyc,thick] {5*x/(1+x)};
        \addplot[domain=0:4,branchout,thick] {3*x/(0.1+x)};
        \addplot[domain=0:4,sumcolor,thick,visible on=<6->] {5*x/(1+x)-3*x/(0.1+x)};
        \addplot[dashed,gray,thick] coordinates {(1.25,0) (1.25,2.77)};
        \node[right,align=left,visible on=<6->] (threetext) at (axis cs:0.05,4.7) {\scriptsize unstable non-zero\\[-0.4em]\scriptsize steady state};
      \end{axis}
     \iftoggle{article} {}
     {
      \node[draw,fit=(plot3) (ylabel3),line width=2pt, fill=none,rounded corners=3pt,opacity=0.2, black!40,visible on=<7->]{};
  }

      \begin{axis}[name=plot4,axis x line=middle,axis y line=left,xlabel near ticks,ylabel near ticks,xmin=0,ymin=-2.5,xmax=2.9,ymax=5.9,xlabel={[$X$]},ylabel={flux},samples=60,at=(phase.right of south east),anchor=left of south west,width=\plotwidth,height=\plotheight,yticklabels={,,},xticklabels={,,},tick label style={major tick length=0pt},ylabel style={name=ylabel4},xshift=1mm,visible on=<5->]%,axis background/.style=kstyle]
        \addplot[domain=0:2.9,autocatacyc,thick] {5*x/(0.2+x)};
        \addplot[domain=0:2.9,branchout,thick] {4*x/(1+x)};
        \addplot[domain=0:2.9,sumcolor,thick,visible on=<6->] {5*x/(0.2+x)-4*x/(1+x)};
        \node[right,align=left,visible on=<6->] (fourtext) at (axis cs:0.05,5) {\scriptsize  no stable steady state};
     \end{axis}
     \iftoggle{article} {}
     {
      \node[draw,fit=(plot4) (ylabel4),line width=2pt, kstyle,fill=none,rounded corners=3pt]{};
     }

        \iftoggle{article} {
          \node [at=(plot2.north west),xshift=-0.6cm,yshift=0.35cm] (B) {(B)};
        }{}
    \end{scope}
  \end{tikzpicture}


    \end{adjustbox}
    \urltext
}

\frame{\frametitle{Conclusions drawn from the simple model apply under various extensions}
\begin{itemize}
    \item Using bisubstrate reaction schemes for the autocatalytic reaction
        \begin{itemize}
            \item Critical lower concentration of the assimilated metabolite exists
            \item Upper bound on the affinity of the branch reaction remains in most schemes
        \end{itemize}
        \pause
    \item Assuming the autocatalytic reaction is reversible
        \begin{itemize}
            \item Relaxes the constraint on the ratio of maximal fluxes between the autocatalytic and the branch reaction
        \end{itemize}
        \pause
    \item Assuming the branch reaction is reversible
        \begin{itemize}
            \item Depending on the consumption of the branch reaction product, either the branch reaction, or the reaction downstream of it must have limited affinity
        \end{itemize}
\end{itemize}
    \urltext
} 

\frame{\frametitle{Directed evolution towards function of the CBB cycle required changes in kinetic parameters of main branch reactions }
\begin{itemize}
    \item 3 Directed evolution repeats evolved functioning CBB cycle
    \item Single common mutation: The major branch reaction gene, PRS
        \begin{itemize}
            \item With other, different mutations in each strain
        \end{itemize}
    \item In all cases $\sfrac{K_{\text{cat}}}{K_M}$ of PRS decreased
    \item Minimal changes required for CBB function include mutations in other major branch reactions
    \end{itemize}
    \urltext
} 

\frame{\frametitle{Why should you care about autocatalytic cycles?}
\begin{itemize}
    \item Key metabolic processes are autocatalytic
    \begin{itemize}
        \item In glycolysis ATP investment is required for the production of ATP
    \end{itemize}
    \item Systematic search reveals autocatalytic cycles are abundant in central carbon metabolism
\end{itemize}
    \urltext
}

\frame{\frametitle{Autocatalytic cycles are abundant in central carbon metabolism}
    \begin{adjustbox}{max totalsize={\textwidth}{0.8\textheight},center}
        \begin{tikzpicture}
 \colorlet{ptsinit}{cyan}
  \colorlet{cbbinit}{yellow}
  \colorlet{glyinit}{magenta}
  \colorlet{fbainit}{cyan}
  \colorlet{ppinit}{magenta}

  \pgfmathsetlength{\assimwidth}{1.5pt};

  \node[metaboliteStyle] (g6p) {g6p};


  \node[metaboliteStyle,below=of g6p.center] (f6p) {f6p};
  \node[metaboliteStyle,below=of f6p] (fbp) {fbp};
  \node[metaboliteStyle,shape=coordinate,below=of fbp.center](fbamid) {};
  \node[metaboliteStyle,below left=of fbamid.center] (dhap) {dhap};
  \node[metaboliteStyle]at([xshift=1.4cm]dhap -|fbamid.center) (gap) {gap};
  \node[metaboliteStyle,below=of gap.center] (bpg) {bpg};
  \node[metaboliteStyle,below=of bpg.center] (3pg) {3pg};
  \node[metaboliteStyle,below=of 3pg.center] (2pg) {2pg};
  \node[metaboliteStyle,below=of 2pg.center] (pep) {pep};
  \node[metaboliteStyle,below=of pep.center] (pyr) {pyr};
  \node[metaboliteStyle,below=of pyr.center,rectangle,draw=assimcol,rounded corners=2pt] (aca) {accoa};
  \node[shape=coordinate,below=of aca] (dummyglta) {};
  \node[metaboliteStyle,left=of dummyglta] (oaa) {oaa};
  \node[metaboliteStyle,right=of dummyglta] (cit) {cit};
  \node[metaboliteStyle,right=of cit] (icit) {icit};
  \node[metaboliteStyle,below=of icit.center] (akg) {akg};
  \node[metaboliteStyle,below=of akg.center] (sca) {sca};
  \node[metaboliteStyle,below=of oaa.center] (mal) {mal};
  \node[metaboliteStyle,below=of mal.center] (fum) {fum};
  \node[metaboliteStyle,right=of mal] (glx) {glx};
  \node[metaboliteStyle,right=of fum] (suc) {suc};
  \node[metaboliteStyle,right=of g6p] (6pgi) {6pgi};
  \node[metaboliteStyle,shape=coordinate,right=of f6p] (s7pspace) {};
  \node[metaboliteStyle,right=of s7pspace] (s7p) {s7p};
  \node[metaboliteStyle,right=of s7p] (r5p) {r5p};
  \node[metaboliteStyle,right=of r5p] (ru5p) {ru5p};
  \node[metaboliteStyle,above=of ru5p.center] (6pgc) {6pgc};
  \node[metaboliteStyle,] at (fbp.center -| s7p.center) (e4p) {e4p};
  \node[metaboliteStyle,] at(e4p.center -| r5p.center) (xu5p) {xu5p};
  \path[] (r5p) -- (gap) coordinate [pos=0.2] (midtkt1) {};
  \path[] (e4p) -- (gap) coordinate [pos=0.4] (midtkt2) {};
  \path[] (e4p) -- (s7pspace) coordinate [pos=0.5] (midtal) {};
  \draw[->] (g6p) -- (f6p);
  \draw[->] ([xshift=0.1cm]f6p.south) -- ([xshift=0.1cm]fbp.north);
  \draw[<-] ([xshift=-0.1cm]f6p.south) -- ([xshift=-0.1cm]fbp.north);
  \draw [->] (fbamid) [out=-90,in=45] to (dhap);
  \draw [->] (fbamid) [out=-90,in=135] to (gap);
  \draw[->] (3pg) -- (2pg);
  \draw[->] (2pg) -- (pep);
  \draw[->] (pep) -- (pyr);
  \draw[<->] (gap) -- (bpg);
  \draw[<->] (bpg) -- (3pg);
  \draw[<-] (ru5p) -- (xu5p);
  \draw[<-] (xu5p) [out=180,in=-90] to (midtkt1);
  \draw[] (midtkt1) [out=90,in=0] to (s7p);
  \draw[<-] (r5p) [out=180,in=90] to (midtkt1);
  \draw[] (midtkt1) [out=270,in=0] to (gap);
  \draw[<-] (e4p) [out=-60,in=0] to (midtkt2);
  \draw[] (midtkt2) [out=180,in=90] to (gap);
  \draw[<-] (xu5p) [out=245,in=0] to (midtkt2);
  \draw[] (midtkt2) [out=180,in=-30] to (f6p);
  \draw[] (midtal) [out=90,in=0] to (f6p);
  \draw[<-] (s7p) [out=180,in=90] to (midtal);
  \draw[] (midtal) [out=-90,in=180] to (e4p);
  \draw[<-] (gap) [out=55,in=-90] to (midtal);
  \draw [<-] (fbp) [out=-90,in=90] to (fbamid);
  \draw [<->] (dhap) -- (gap);
  \draw[->] (pyr) -- (aca);
  \draw[->] (oaa) -- (cit) node [pos=0.9] (midglta) {};
  \draw [] (aca) [out=-70,in=180] to (midglta);
  \draw[->] (cit) -- (icit);
  \draw[->] (icit) -- (suc) node [pos=0.3] (midacea) {};
  \draw[->] (midacea) [out=220,in=0] to (glx);
  \draw[->] (icit) -- (akg);
  \draw[->] (akg) -- (sca);
  \draw[->] (sca) -- (suc);
  \draw[->] (suc) -- (fum);
  \draw[->] (fum) -- (mal);
  \draw[->] (glx) -- (mal) node [pos=0.9] (midaceb) {};
  \draw[->] (mal) -- (oaa);
  \draw[] (aca) [out=-90,in=0] to (midaceb);
  \draw[->] (g6p) -- (6pgi);
  \draw[->] (6pgi) -- (6pgc);
  \draw[->] (6pgc) -- (ru5p);
  \draw[<-] (ru5p) -- (r5p);
  %%%% upper pts
  \node[shape=coordinate,left=12mm of g6p.center] (ptsmid) {};
  \node[metaboliteStyle,shift={(-7mm,-7mm)},gray] at (g6p.center) (pyr1) {pyr};
  \node[metaboliteStyle,left=7mm of ptsmid,rectangle,draw=assimcol,rounded corners=2pt] (gluc) {gluc};
  \draw[] (gluc) -- (ptsmid);
  \draw[->] (ptsmid) [out=0,in=90] to (pyr1);
  \node[shape=coordinate,left=2.5cm of pep.center] (pts3) {};
  \draw[] (pep) [out=180,in=0] to (pts3);
  \node[shape=coordinate,left=2.5cm of pyr.center] (pts5) {};
  \node[shape=coordinate,left=2.1cm of f6p.center] (ptstop) {};
  \node[shape=coordinate] at(ptstop |- 2pg.center) (ptsbottom) {};
  \draw[] (pts3) [in=-90,out=180] to (ptsbottom);
  \draw[] (ptsbottom) [in=-90,out=90] to (ptstop);
  \draw[] (ptstop) [in=180,out=90] to (ptsmid);
  \draw[->] (ptsmid) -- (g6p);
  \node[shape=coordinate,shift={(-\highlightrad,-\highlightrad)}] at (pep.south -| ptsbottom) (ptsbottomlimit) {};
  \node[shape=coordinate,shift={(-\highlightrad,\highlightrad)}] at (g6p.north -| ptstop) (ptstoplimit) {};
  \draw[assimcol,line width=\assimwidth] (gluc) -- (ptsmid);
  \node[metaboliteStyle,] at(3pg.center -| ru5p.center) (rub) {rubp};
  \node[metaboliteStyle,rectangle,draw=assimcol,rounded corners=2pt] at(2pg -| xu5p.center) (co2) {\ce{CO2}};
  \draw[->] (ru5p) [out=-90,in=90] to (rub);
  \draw[->] (rub) -- (3pg) coordinate [pos=0.9] (rubisco);
  \draw[assimcol,line width=\assimwidth] (co2) [out=90,in=0] to (rubisco);
  \draw [assimcol,line width=\assimwidth] (aca) [out=-70,in=180] to (midglta);
  \draw[assimcol,line width=\assimwidth] (aca) [out=-90,in=0] to (midaceb);
  \draw[opacity=0.2,fill=ptsinit,rounded corners=\highlightrad,visible on=<3->] ([shift={(\highlightrad,\highlightrad)}]g6p.north east) -- ([xshift=\highlightrad] fbp.east) -- ([shift={(\highlightrad,\highlightrad)}]gap.north east)--([shift={(\highlightrad,-\highlightrad)}]pep.south east) -- (ptsbottomlimit) -- node[midway] (ptsshademid) {} ([yshift=-1.2cm]ptstoplimit) -- ([shift={(-1mm,\highlightrad)}]g6p.north -| ptsmid) -- cycle;

  \draw[very thick,dashed,cyan,->,visible on=<3->] (ptsshademid) -- ++(-1.5cm,0cm); 

  \draw[opacity=0.2,fill=glyinit,rounded corners=\highlightrad,visible on=<4->] ([shift={(-\highlightrad,2*\highlightrad)}]oaa.west) -- ([shift={(\highlightrad,2*\highlightrad)}]icit.east) -- node[midway] (glyshadedmid) {}([shift={(\highlightrad,-0.5*\highlightrad)}]icit.south east) -- ([shift={(0.5*\highlightrad,-2*\highlightrad)}]suc.east) -- ([shift={(-\highlightrad,-2*\highlightrad)}]fum.west) -- cycle;

  \draw[very thick,dashed,magenta,->,visible on=<4->] (glyshadedmid) -- ++(1.5cm,0cm); 

  \node[shape=coordinate] at (dhap.south -| gap.west) (cbbmid) {};
  \draw[opacity=0.2,fill=cbbinit,rounded corners=\highlightrad,visible on=<2->] ([shift={(-\highlightrad,2.2*\highlightrad)}]f6p.west) -- ([shift={(3*\highlightrad,2.2*\highlightrad)}]ru5p.center) -- node[midway] (cbbshadedmid) {} ([shift={(3*\highlightrad,-2*\highlightrad)}]rub.center) -- ([shift={(-\highlightrad,-2*\highlightrad)}]3pg.west) -- ([shift={(-\highlightrad,-\highlightrad)}]cbbmid) -- ([shift={(-0.5*\highlightrad,-\highlightrad)}]dhap.south west) -- ([shift={(-0.5*\highlightrad,0.5*\highlightrad)}]dhap.north west) -- ([xshift=-\highlightrad]fbp.west) -- cycle;

  \draw[very thick,dashed,cbbinit,->,visible on=<2->] (cbbshadedmid) -- ++(1.5cm,0cm); 

  %% CBB cycle
  \begin{scope} [shift={(11.2cm,-4.5cm)},radius=2cm,visible on=<2->]
    \draw[lightgray,rounded corners=\highlightrad] (-2.7,-2.1) rectangle +(5.5,4.5);

    \iftoggle{article} {
        \node at (-2.2cm,2cm) (I) {\large  \textbf{I}};
    }{ }
    \node[anchor=north] at(0cm,-2.2cm) (cbbreac) {{\fontfamily{cmss}\selectfont 5}  rubp + {\fontfamily{cmss}\selectfont 5} \ce{CO2} $\rightarrow$ {\fontfamily{cmss}\selectfont 6} rubp};

    \preassim{\arcwidth}{-95}{-270}{\autocatalrad}{cbbinit}
    \postassim{\arcwidth}{90}{-45}{\autocatalrad}{cbbinit}{6/5}
    \assim{\arcwidth}{90}{-30}{\autocatalrad}{6/5}
    \arrowhead{\arcwidth}{-45}{\autocatalrad}{cbbinit}

    \node at (-37:\autocatalrad+3.5*\arcwidth) (rubptwo) {+rubp};
    \node at(125:\autocatalrad+2.3*\arcwidth) (cotwo) {5 \ce{CO2}};
    \node at (-67:\autocatalrad) (3pgc) {5 rubp};
  \end{scope}

  %glyoxilate cycle
\begin{scope} [shift={(8.9cm,-13.5cm)},radius=2cm,visible on=<4->]
  \draw[lightgray,rounded corners=\highlightrad] (-2.4cm,-2.1cm) rectangle +(5.1,4.6);

  \iftoggle{article} {
      \node at (-2cm,2.2cm) (II) {\large  \textbf {II}};
  }{ }

  \node[anchor=north] at(0cm,-2.2cm) (glyreac) {mal + {\fontfamily{cmss}\selectfont 2} accoa $\rightarrow$ {\fontfamily{cmss}\selectfont 2} mal};
  
    \preassim{\arcwidth}{-90}{-270}{\autocatalrad}{glyinit}
    \postassim{\arcwidth}{90}{-45}{\autocatalrad}{glyinit}{2}
    \assim{\arcwidth}{90}{-30}{\autocatalrad}{2}
    \arrowhead{\arcwidth}{-45}{\autocatalrad}{glyinit}

    \node at (-37:\autocatalrad+3.5*\arcwidth) (glytwo) {+mal};
    \node at(130:\autocatalrad+2.5*\arcwidth) (acac) {2 accoa};
    \node at (-70:\autocatalrad) (malc) {mal};
  \end{scope}


  %pts cycle
\begin{scope} [shift={(-6.5cm,-5.5cm)},radius=2cm,visible on=<3->]
  \draw[lightgray,rounded corners=\highlightrad] (-2.5,-2) rectangle +(5.2,4.6);

  \iftoggle{article} {
      \node at (-2cm,2.2cm) (III) {\large  \textbf {III}};
  }{ }

  \node[anchor=north] at(-0cm,-2.1cm) (ptsreac) {pep + gluc $\rightarrow$ {\fontfamily{cmss}\selectfont 2} pep + pyr};
    \colorlet{ptsmed}{blue}
    \colorlet{ptsext}{assimcol}

    \pgfmathsetlength{\ptsierad}{\autocatalrad*0.5};
    \pgfmathsetlength{\ptsimrad}{\autocatalrad-\arcwidth};

    \assim{\arcwidth}{100}{-30}{\ptsimrad}{3}
    \arrowhead{\arcwidth}{-45}{\autocatalrad}{ptsinit}

    \pgfmathsetlength{\ptsesrad}{\autocatalrad+\arcwidth/2};
    \pgfmathsetmacro{\startbranch}{-15}

    \draw[color=autocatacyc,line width=3*\arcwidth] (100:\autocatalrad) arc(100:40:\autocatalrad);

    \draw[color=ptsinit,line width=\arcwidth] (\startbranch:\autocatalrad) arc(\startbranch:-45:\autocatalrad);

    \pgfmathsetlength{\ptsarcwidth}{\autocatalrad-\arcwidth};

    \shadedarc[\arcwidth]{-90}{-260}{\autocatalrad}{\ptsarcwidth}{autocatacyc}{ptsinit}

    \colorgradarc{2*\arcwidth}{40}{\startbranch}{\ptsesrad}{autocatacyc}{ptsinit}

    \shadedarc[\arcwidth]{40}{-15}{\ptsarcwidth}{\ptsarcwidth-\arcwidth}{white}{autocatacyc}

    \begin{scope}[shift={(\startbranch:2*\autocatalrad+\arcwidth)}]
        \draw[color=ptsinit,line width=\arcwidth] (\startbranch+180:\autocatalrad) arc (\startbranch+180:\startbranch+180+30:\autocatalrad);
        \revarrowhead{\arcwidth}{195}{\autocatalrad}{ptsinit}
    \end{scope}

    \node at (-42:\autocatalrad+3.3*\arcwidth) (peptwo) {+pep};

    \node at (-70:\autocatalrad) (pepc) {pep};
    \node[color=gray] at (-30:\ptsierad) (pyrc) {pyr};
    \node at(130:\autocatalrad+2.5*\arcwidth) (acac) {gluc};
  \end{scope}

\end{tikzpicture}


    \end{adjustbox}
    \urltext
}

\frame{\frametitle{Stability criteria of the simple model can be extended for complex cycles}
    \begin{adjustbox}{max totalsize={\textwidth}{0.6\textheight},center}
      \begin{tikzpicture}[>=latex']
    \iftoggle{poster} {}
    {
    \iftoggle{article} {
    \begin{scope}[shift={(-5cm,0cm)},node distance = 2cm]
        \node (X1) {$X_1$};
        \node[left=of X1]  (X2) {$X_2$};
        \draw [->,line width=1pt,autocatacyc] (X1.south) [out=-90,in=-90] to  node [pos=0.5,above] (fa1) {$f_{a_1}$} (X2.south);
        \draw [->,line width=1pt,autocatacyc] (X2.north) [out=90,in=90] to node [pos=0.5,shape=coordinate,yshift=0.5pt] (assimpt) {} node [pos=0.6,above] (fa2) {$f_{a_2}$} (X1.north);
        \draw [line width=1pt,assimcol] (assimpt) arc (-90:-140:0.7cm) node [pos=1,above] (e) {$A$};
        \draw [->,line width=1pt,branchout] ([xshift=0.1cm]X1.south) arc (190:270:1cm) node [pos=0.3,right,xshift=1mm] {$f_{b_1}$};
        \draw [->,line width=1pt,branchout] ([xshift=-0.1cm]X2.north) arc (10:90:1cm) node [pos=0.3,left] {$f_{b_2}$};
    \end{scope}}
    {}
}
    \begin{scope}[shift={(0cm,0cm)},node distance = 1cm]
        \node (X1) {$X_1$};
        \node[below left=of X1]  (X2) {$X_2$};
        \node[above left=of X1]  (Xn) {$X_n$};
        \draw [->,line width=1pt,autocatacyc] (X1.south) [out=-90,in=0] to  node [pos=0.7,right,xshift=1mm] (fa1) {$f_{a_1}$} (X2.east);
        \draw [->,line width=1pt,autocatacyc] (Xn.east) [out=0,in=90] to node [pos=0.4,shape=coordinate] (assimpt) {} node [pos=0.4,right,xshift=1mm] (fan) {$f_{a_n}$}(X1.north);
        \draw [line width=1pt,assimcol] (assimpt) arc (-120:-170:0.7cm) node [pos=1,above] (e) {$A$};
        \draw [->,line width=1pt,branchout] ([xshift=0.1cm]X1.south) arc (190:270:1cm) node [pos=0.3,right,xshift=1mm] {$f_{b_1}$};
        \draw [->,line width=1pt,branchout] ([yshift=-0.5mm]X2.west) arc (90:180:1cm) node [pos=0.3,below,xshift=1mm] {$f_{b_2}$};
        \draw [->,line width=1pt,branchout] (Xn.north) arc (10:90:1cm) node [pos=0.3,left] {$f_{b_n}$};
        \draw [->,line width=1pt,autocatacyc,dashed] ([yshift=0.5mm]X2.west) [out=180,in=-90] to ($(X1)+(-3,0)$) node [right] (fmid) {$f_{a_2}\dots f_{a_{n-1}}$} to [out=90,in=180] (Xn.west);
    \end{scope}
    \iftoggle{poster} {}
    {
    \iftoggle{article} {
        \node [shift={(-8cm,3cm)}] (A) {(A)};
        \node [right of=A,xshift=4cm] (B) {(B)};
    }{}
}
\end{tikzpicture}


    \end{adjustbox}
    \pause
    \begin{itemize}
        \item At steady state: $\sum f_{b_i}=f_{a_n}$
        \pause
    \item Sufficient condition for stability is: $\exists_i \quad \beta_i \geq \alpha_i$
        
        where $\beta_i=\frac{df_{b_i}}{dX_i}\Big\rvert_{X_i^*}$ and $\alpha_i=\frac{df_{a_i}}{dX_i}\Big\rvert_{X_i^*}$
    \end{itemize}
    \urltext
}

\frame{\frametitle{Theoretical $\beta_i\geq\alpha_i$ constraint results in experimental prediction on reaction saturation level}
    \begin{itemize}
        \item Reaction saturation  is the ratio of the actual flux to the potential flux, given expression level and catalytic rate
            \pause
        \item For monotonically increasing, bounded, concave functions: saturation and derivative are inversely correlated
            \pause
        \item Therefore, $\beta_i\geq\alpha_i$ imply that branch reaction is less saturated than autocatalytic reaction
    \end{itemize}
    \urltext
}

\frame{\frametitle{Analysis of experimental fluxomics data\footnote{Gerosa et. al., Cell Systems 2015} and proteomics data\footnote{Schmidt et. al., Nature Biotechnology 2016} shows branch reactions are consistently less saturated than autocatalytic reactions}
    \begin{adjustbox}{max totalsize={\textwidth}{0.69\textheight},center}
      \begin{tikzpicture}
  \tikzset{
    figArrowStyle/.style={arrows={-{Stealth[inset=0pt,scale=#1,angle'=60]}}},
    figArrowStyle/.default=0.25
  }
  \tikzset{
    capArrowStyle/.style={arrows={-{Stealth[inset=0pt,scale=0.25,angle'=60,color=#1]}}},
    capArrowStyle/.default=autocatacycfl
    }

  \tikzset{
    ratioRect/.style={rectangle,fill=graybg,rounded corners=2pt}}

  \newcommand{\coloredRatio}[2]}{\mathbf{\color{autocatacyc}#2\%}}$}}
  }

\iftoggle{article} {
    \pgfmathsetlength{\nodedist}{1cm}
    \renewcommand{\fontsizedef}{\normalsize}
    \renewcommand{\ratiosizedef}{\Large}
    \renewcommand{\coloredRatio}[2]}{\mathbf{\color{autocatacyc}#2\%}}$}}
  }
}{
    \only<3-> {
        \pgfmathsetlength{\nodedist}{1cm}
        \renewcommand{\fontsizedef}{\normalsize}
        \renewcommand{\ratiosizedef}{\Large}
        \renewcommand{\coloredRatio}[2]}{\mathbf{\color{autocatacyc}##2\%}}$}}
      }
    }
}

    \pgfmathsetlength{\headlinedist}{0.3cm}
    %prediction
  \begin{scope}[shift={(-3cm,0cm)}]
      \node[ratioRect,align=center] (prediction) {\textbf{Prediction:} $\mathbf{\color{branchout}XX\%} < \mathbf{\color{autocatacyc}YY\%}$ \\ for at least one branch reaction};
  \end{scope}
  %Galactose
  \begin{scope}[shift={(5cm,-7.5cm)},visible on=<3->,font=\fontsizedef]
    \def\galmaxflux{0.5mm}
    \def\galglt{1.52*\galmaxflux}
    \def\galcapvalglt{20}
    \def\galcapglt{\galglt/\galcapvalglt*100}
    \def\galacn{1.52*\galmaxflux*1.5}
    \def\galacea{1.02*\galmaxflux}
    \def\galcapvalacea{30}
    \def\galcapacea{\galacea/\galcapvalacea*100}
    \def\galaceb{1.02*\galmaxflux}
    \def\galsdh{1.26*\galmaxflux}
    \def\galfum{1.26*\galmaxflux}
    \def\galmdh{2.28*\galmaxflux}
    \def\galpck{0.85*\galmaxflux}
    \def\galcapvalpck{25}
    \def\galcappck{\galpck/\galcapvalpck*100}
    \def\galicd{0.5*\galmaxflux}
    \def\galcapvalicd{10}
    \def\galcapicd{\galicd/\galcapvalicd*100}

    \node[] (galactose) {\textbf{Galactose input}};
    \node[metaboliteStyle,inputcol,below=\headlinedist of galactose] (aca) {aca};
    \node[shape=coordinate,below=of aca] (dummyglta) {};
    \node[metaboliteStyle,left=of dummyglta] (oaa) {oaa};
    \node[metaboliteStyle] at (aca -| oaa) (pep) {pep};
    \node[metaboliteStyle,right=of dummyglta] (cit) {cit};
    \node[metaboliteStyle,right=of cit] (icit) {icit};
    \node[metaboliteStyle,below=of icit.center] (akg) {akg};
    \node[metaboliteStyle,below=of akg.center] (sca) {sca};
    \node[metaboliteStyle,below=of oaa.center] (mal) {mal};
    \node[metaboliteStyle,below=of mal.center] (fum) {fum};
    \node[metaboliteStyle,right=of mal] (glx) {glx};
    \node[metaboliteStyle,right=of fum] (suc) {suc};
    \path[] (oaa) -- (cit) node [pos=0.85,shape=coordinate] (midglta) {};
    \draw[line width=\galcapglt,autocatacyc!40] ([yshift=-0.35*\galglt]oaa.east) -- ([yshift=-0.35*\galglt]midglta) ;

    \node[anchor=east] at (oaa.west) (galb1) {\coloredRatio{\galcapvalpck}{\galcapvalglt}};
    \node[ratioRect,anchor=south] at (icit.north) (galb2) {\coloredRatio{\galcapvalicd}{\galcapvalacea}};

    \draw[line width=\galglt,autocatacycfl] ([yshift=-0.35*\galglt]oaa.east) -- ([yshift=-0.35*\galglt]midglta);
    \draw[figArrowStyle,line width=\galglt*1.5,autocatacycfl] (midglta) -- (cit);
    \draw[capArrowStyle=branchoutfl,line width=\galcappck,branchout!40] (oaa) -- (pep);
    \draw[figArrowStyle,line width=\galpck,branchoutfl] (oaa) -- (pep);
    \draw [inputcol,line width=0.5*\galglt] (aca) [out=-70,in=180] to ([yshift=0.35*\galglt]midglta);
    \draw[figArrowStyle,line width=\galacn,autocatacycfl] (cit) -- (icit);
    \path[] (icit.south west) -- (suc) node [pos=0.3,shape=coordinate] (midacea) {};
    \draw[line width=\galcapacea*1.5,autocatacyc!40] (icit.south west) -- (midacea);
    \draw[capArrowStyle,line width=\galcapacea,autocatacyc!40] ([xshift=\galcapacea*0.35]midacea) -- ([xshift=\galcapacea*0.4]suc);
    \draw[capArrowStyle,line width=\galcapacea,autocatacyc!40] ([xshift=\galcapacea*0.38]midacea) -- ([xshift=\galcapacea*0.4]suc);
    \draw[capArrowStyle,line width=\galcapacea*0.5,autocatacyc!40] ([shift={(-\galcapacea*0.35,\galcapacea*0.35)}]midacea) [out=220,in=0] to (glx);
    \draw[capArrowStyle,line width=\galcapacea*0.5,autocatacyc!40] ([shift={(-\galacea*0.35,\galacea*0.35)}]midacea) [out=220,in=0] to (glx);
    \draw[line width=\galacea*1.5,autocatacycfl] (icit.south west) -- (midacea);
    \draw[figArrowStyle,line width=\galacea,autocatacycfl] ([xshift=\galacea*0.4]midacea) -- ([xshift=\galacea*0.4]suc);
    \draw[figArrowStyle,line width=\galacea*0.5,autocatacycfl] ([shift={(-\galacea*0.35,\galacea*0.35)}]midacea) [out=220,in=0] to (glx);
    \draw[capArrowStyle=branchoutfl,line width=\galcapicd*1.5,branchout!40] (icit) -- (akg);
    \draw[figArrowStyle,line width=\galicd*1.5,branchoutfl] (icit) -- (akg);
    \draw[->] (akg) -- (sca);
    \draw[->] (sca) -- (suc);
    \draw[figArrowStyle,line width=\galsdh,autocatacycfl] (suc) -- (fum);
    \draw[figArrowStyle,line width=\galfum,autocatacycfl](fum) -- (mal);
    \path[] (glx) -- (mal) node [pos=0.85,shape=coordinate] (midaceb) {};
    \draw[line width=\galaceb*0.5,autocatacycfl] ([yshift=-0.25*\galaceb]glx) -- ([yshift=-0.25*\galaceb]midaceb);
    \draw[inputcol,line width=0.5*\galaceb] ([xshift=-0.5mm]aca.south) [out=-90,in=0] to ([yshift=0.25*\galaceb]midaceb);
    \draw[figArrowStyle,line width=\galaceb,autocatacycfl] (midaceb) -- (mal);
    \draw[figArrowStyle,line width=\galmdh,autocatacycfl] (mal) -- (oaa);
  \end{scope}

  %acetate
  \begin{scope}[shift={(5cm,-1.5cm)},node distance=\nodedist,font=\fontsizedef]
    \def\acemaxflux{0.2mm}
    \def\aceglt{8.83*\acemaxflux}
    \def\acecapvalglt{70}
    \def\acecapglt{\aceglt/\acecapvalglt*100}
    \def\aceacn{8.83*\acemaxflux*1.5}
    \def\aceacea{4.14*\acemaxflux}
    \def\acecapvalacea{100}
    \def\acecapacea{\aceacea}
    \def\aceaceb{4.14*\acemaxflux}
    \def\acesdh{8.4*\acemaxflux}
    \def\acefum{8.4*\acemaxflux}
    \def\acemdh{10.67*\acemaxflux}
    \def\acecapvalmdh{100}
    \def\acemae{1.87*\acemaxflux}
    \def\acecapvalmae{15}
    \def\acecapmae{\acemae/\acecapvalmae*100}
    \def\acepck{3.11*\acemaxflux}
    \def\acecapvalpck{75}
    \def\acecappck{\acepck/\acecapvalpck*100}
    \def\aceicd{4.7*\acemaxflux}
    \def\acecapvalicd{65}
    \def\acecapicd{\aceicd/\acecapvalicd*100}

    \node[] (acetate) {\textbf{Acetate input}};
    \node[metaboliteStyle,inputcol,below=\headlinedist of acetate] (aca) {aca};
    \node[shape=coordinate,below=of aca] (dummyglta) {};
    \node[metaboliteStyle,left=of dummyglta] (oaa) {oaa};
    \node[metaboliteStyle] at (aca -| oaa) (pep) {pep};
    \node[metaboliteStyle,right=of dummyglta] (cit) {cit};
    \node[metaboliteStyle,right=of cit] (icit) {icit};
    \node[metaboliteStyle,below=of icit.center] (akg) {akg};
    \node[metaboliteStyle,below=of akg.center] (sca) {sca};
    \node[metaboliteStyle,below=of oaa.center] (mal) {mal};
    \node[metaboliteStyle,below=of mal.center] (fum) {fum};
    \node[metaboliteStyle,right=of mal] (glx) {glx};
    \node[metaboliteStyle,left=of mal] (pyr) {pyr};
    \node[metaboliteStyle,right=of fum] (suc) {suc};
    \path[] (oaa) -- (cit) node [pos=0.85,shape=coordinate] (midglta) {};
    \draw[line width=\acecapglt,autocatacyc!40] ([yshift=-0.25*\aceglt]oaa.east) -- ([yshift=-0.25*\aceglt]midglta);
    \node[anchor=south east,visible on=<2->] at (oaa.north west) (aceb1) {\coloredRatio{\acecapvalpck}{\acecapvalglt}};
    \draw[line width=\aceglt,autocatacycfl] ([yshift=-0.25*\aceglt]oaa.east) -- ([yshift=-0.25*\aceglt]midglta);
    \draw[figArrowStyle,line width=\aceglt*1.5,autocatacycfl] (midglta) -- (cit);
    \draw[capArrowStyle=branchoutfl,line width=\acecappck,branchout!40] (oaa) -- (pep);
    \draw[figArrowStyle,line width=\acepck,branchoutfl] (oaa) -- (pep);
    \draw[capArrowStyle=branchoutfl,line width=\acecapmae,branchout!40] (mal) -- (pyr);
    \node[ratioRect,anchor=south east,visible on=<2->] at (mal.north west) (aceb2) {\coloredRatio{\acecapvalmae}{\acecapvalmdh}};
    \draw[figArrowStyle,line width=\acemae,branchoutfl] (mal) -- (pyr);
    \draw [inputcol,line width=0.5*\aceglt] (aca) [out=-70,in=180] to ([yshift=0.4*\aceglt]midglta);
    \draw[figArrowStyle,line width=\aceacn,autocatacycfl] (cit) -- (icit);
    \path[] (icit.south west) -- (suc) node [pos=0.3,shape=coordinate] (midacea) {};
    \draw[line width=\acecapacea*1.5,autocatacyc!40] (icit.south west) -- (midacea);
    \draw[capArrowStyle,line width=\acecapacea,autocatacycfl!40] ([xshift=\acecapacea*0.35]midacea) -- ([xshift=\acecapacea*0.4]suc);
    \node[ratioRect,anchor=south,visible on=<2->] at (icit.north) (aceb3) {\coloredRatio{\acecapvalicd}{\acecapvalacea}};
    \draw[capArrowStyle,line width=\acecapacea*0.5,autocatacyc!40] ([shift={(-\acecapacea*0.35,\acecapacea*0.35)}]midacea) [out=220,in=0] to (glx);
    \draw[line width=\aceacea*1.5,autocatacycfl] (icit.south west) -- (midacea);
    \draw[figArrowStyle,line width=\aceacea,autocatacycfl] ([xshift=\aceacea*0.4]midacea) -- ([xshift=\aceacea*0.4]suc);
    \draw[figArrowStyle,line width=\aceacea*0.5,autocatacycfl] ([shift={(-\aceacea*0.35,\aceacea*0.35)}]midacea) [out=220,in=0] to (glx);
    \draw[capArrowStyle=branchoutfl,line width=\acecapicd*1.5,branchout!40] (icit) -- (akg);
    \draw[figArrowStyle,line width=\aceicd*1.5,branchoutfl] (icit) -- (akg);
    \draw[->] (akg) -- (sca);
    \draw[->] (sca) -- (suc);
    \draw[figArrowStyle,line width=\acesdh,autocatacycfl] (suc) -- (fum);
    \draw[figArrowStyle,line width=\acefum,autocatacycfl](fum) -- (mal);
    \path[] (glx) -- (mal) node [pos=0.85,shape=coordinate] (midaceb) {};
    \draw[line width=\aceaceb*0.5,autocatacycfl] ([yshift=-0.25*\aceaceb]glx) -- ([yshift=-0.25*\aceaceb]midaceb);
    \draw[inputcol,line width=0.5*\aceaceb] ([xshift=-0.5mm]aca.south) [out=-90,in=0] to ([yshift=0.25*\aceaceb]midaceb);
    \draw[figArrowStyle,line width=\aceaceb,autocatacycfl] (midaceb) -- (mal);
    \draw[figArrowStyle,line width=\acemdh,autocatacycfl] (mal) -- (oaa);
 
  \end{scope}

  %Glucose
  \begin{scope}[shift={(-9cm,-0.2cm)},visible on=<3->,font=\fontsizedef]
    \def\glucmaxflux{0.35mm}
    \def\glucpgi{5.7*\glucmaxflux}
    \def\gluccapvalpgi{75}
    \def\gluccappgi{\glucpgi/\gluccapvalpgi*100}
    \def\glucpfk{7.06*\glucmaxflux}
    \def\glucfba{7.06*\glucmaxflux}
    \def\gluctpi{7.06/2*\glucmaxflux}
    \def\glucgap{15.71/2*\glucmaxflux}
    \def\glucpgk{15.71/2*\glucmaxflux}
    \def\glucgpm{14.56/2*\glucmaxflux}
    \def\gluceno{14.56/2*\glucmaxflux}
    \def\glucpyk{2.49/2*\glucmaxflux}
    \def\gluccapvalpyk{35}
    \def\gluccappyk{\glucpyk/\gluccapvalpyk*100}
    \def\glucppc{2.45/2*\glucmaxflux}
    \def\gluccapvalppc{45}
    \def\gluccapppc{\glucppc/\gluccapvalppc*100}
    \def\gluczwf{3.92*\glucmaxflux}
    \def\gluccapvalzwf{55}
    \def\gluccapzwf{\gluczwf/\gluccapvalzwf*100}
    \def\glucpts{9.65*\glucmaxflux}
    \def\gluccapvalpts{100}
    %\def\gluccapvalcomp{\pgfmathparse{round((\glucppc+\glucpyk)/(\gluccapppc+\gluccappyk)*20)*5}\pgfmathprintnumber{\pgfmathresult}}
    \def\gluccapvalcomp{\pgfmathparse{round(0.43*20)*5}\pgfmathprintnumber{\pgfmathresult}}
    \def\gluccappts{\glucpts}

    \node[] (glucose) {\textbf{Glucose input}};
    \node[metaboliteStyle,below=\headlinedist of glucose] (g6p) {g6p};
    \node[metaboliteStyle,below=of g6p.center] (f6p) {f6p};
    %%% ptstop
    \node[shape=coordinate,left=1.1cm of g6p] (ptsmid) {};
    \node[metaboliteStyle,shift={(-11mm,2mm)},gray] at (f6p) (pyr1) {pyr};
    \node[metaboliteStyle,inputcol,left=of ptsmid] (gluc) {gluc};

    \node[metaboliteStyle,below=of f6p] (fbp) {fbp};
    \node[shape=coordinate,below=of fbp.center](fbamid) {};
    \node[metaboliteStyle,below left=of fbamid.center] (dhap) {dhap};
    \node[metaboliteStyle,below right=of fbamid] (gap) {gap};
    \node[metaboliteStyle,below=of gap.center] (bpg) {bpg};
    \node[metaboliteStyle,below=of bpg.center] (3pg) {3pg};
    \node[metaboliteStyle,below=of 3pg.center] (2pg) {2pg};
    \node[metaboliteStyle,below=of 2pg.center] (pep) {pep};
    \node[metaboliteStyle,right=of pep] (oaa) {oaa};
    \node[metaboliteStyle,below=of pep.center] (pyr) {pyr};
    \node[metaboliteStyle,right=of g6p] (6pgi) {6pgi};
    \draw[capArrowStyle,line width=\gluccappgi,autocatacycfl!40] (g6p) -- (f6p);
    \draw[figArrowStyle,line width=\glucpgi,autocatacycfl] (g6p) -- (f6p);
    \draw[figArrowStyle,line width=\glucpfk,autocatacycfl] (f6p.south) -- (fbp.north);
    \draw [line width=\glucfba,autocatacycfl] (fbp) [out=-90,in=90] to (fbamid);
    \draw [figArrowStyle,line width=\glucfba/2,autocatacycfl] ([xshift=-\glucfba/4]fbamid) [out=-90,in=45] to (dhap);
    \draw [figArrowStyle,line width=\glucfba/2,autocatacycfl] ([xshift=\glucfba/4]fbamid) [out=-90,in=135] to (gap);
    \draw [figArrowStyle,line width=\gluctpi,autocatacycfl] (dhap) -- (gap);

    \draw[figArrowStyle,line width=\glucgap,autocatacycfl] (gap) -- (bpg);
    \draw[figArrowStyle,line width=\glucpgk,autocatacycfl] (bpg) -- (3pg);
    \draw[figArrowStyle,line width=\glucgpm,autocatacycfl] (3pg) -- (2pg);
    \draw[figArrowStyle,line width=\gluceno,autocatacycfl] (2pg) -- (pep);
    \draw[capArrowStyle=branchoutfl,line width=\gluccappyk,branchout!40] (pep) -- (pyr);
    \draw[figArrowStyle,line width=\glucpyk,branchoutfl] (pep) -- (pyr);
    \draw[capArrowStyle=branchoutfl,line width=\gluccapppc,branchout!40] (pep) -- (oaa);
    \node[ratioRect,anchor=south east] at (pep.north west) (glucb2) {\coloredRatio{\gluccapvalcomp}{\gluccapvalpts}};
    \draw[figArrowStyle,line width=\glucppc,branchoutfl] (pep) -- (oaa);
    \draw[capArrowStyle=branchoutfl,line width=\gluccapzwf,branchout!40] (g6p) -- (6pgi);
    \draw[figArrowStyle,line width=\gluczwf,branchoutfl] (g6p) -- (6pgi);
    \node[ratioRect,anchor=north west] at (g6p.south east) (glucb1) {\coloredRatio{\gluccapvalzwf}{\gluccapvalpgi}};
    \node[shape=coordinate,left=2.5cm of pep.center] (pts3) {};
    \draw[line width=\glucpts/2,autocatacycfl] (pep.west) -- (pts3);
    \draw[inputcol,line width=\glucpts] (gluc) -- (ptsmid);
    \node[shape=coordinate,left=2.5cm of pyr.center] (pts5) {};
    \draw[figArrowStyle,line width=\glucpts/2,autocataby] ([yshift=-3/4*\glucpts]ptsmid) [out=0,in=90] to (pyr1);
    \node[shape=coordinate,xshift=-2mm] at (f6p.center -| dhap.west) (ptstop) {};
    \node[shape=coordinate] at(ptstop |- 2pg.center) (ptsbottom) {};
    \draw[line width=\glucpts/2,autocatacycfl] (pts3) [in=-90,out=180] to (ptsbottom);
    \draw[line width=\glucpts/2,autocatacycfl] (ptsbottom) -- (ptstop);
    \draw[line width=\glucpts/2,autocatacycfl] (ptstop) [in=180,out=90] to ([yshift=-3/4*\glucpts]ptsmid);
    \draw[figArrowStyle,line width=\glucpts,autocatacycfl] (ptsmid) -- (g6p);
  \end{scope}

  %Fructose
  \begin{scope}[shift={(-2cm,-1.5cm)},visible on=<3->,font=\fontsizedef]
    \def\frucmaxflux{0.35mm}
    \def\frucfbp{2.46*\frucmaxflux}
    \def\fruccapvalfbp{100}
    \def\fruccapfbp{\frucfbp}
    \def\frucfba{5.87*\frucmaxflux}
    \def\fruccapvalfba{70}
    \def\fruccapfba{\frucfba/\fruccapvalfba*100}
    \def\fructpi{5.87/2*\frucmaxflux}
    \def\frucgap{13.46/2*\frucmaxflux}
    \def\frucpgk{13.46/2*\frucmaxflux}
    \def\frucgpm{12.6/2*\frucmaxflux}
    \def\fruceno{12.6/2*\frucmaxflux}
    \def\frucpyk{0.67/2*\frucmaxflux}
    \def\fruccapvalpyk{5}
    \def\fruccappyk{\frucpyk/\fruccapvalpyk*100}
    \def\frucppc{3.55/2*\frucmaxflux}
    \def\fruccapvalppc{50}
    \def\fruccapppc{\frucppc/\fruccapvalppc*100}
    \def\frucpts{8.33*\frucmaxflux}
    \def\fruccapvalpts{100}
    \def\fruccappts{\frucpts}
    %\def\fruccapvalcomp{\pgfmathparse{round((\frucppc+\frucpyk)/(\fruccapppc+\fruccappyk)*20)*5}\pgfmathprintnumber{\pgfmathresult}}
    \def\fruccapvalcomp{\pgfmathparse{round(0.28*20)*5}\pgfmathprintnumber{\pgfmathresult}}


    \node[] (fructose) {\textbf{Fructose input}};
    \node[metaboliteStyle,below=\headlinedist of fructose] (f6p) {f6p};
    \node[metaboliteStyle,below=of f6p] (fbp) {fbp};
    \node[shape=coordinate,below=of fbp.center](fbamid) {};
    %%% ptstop
    \node[shape=coordinate,left=1.1cm of fbp] (ptsmid) {};
    \node[metaboliteStyle,shift={(-11mm,0mm)},gray] at (fbamid) (pyr1) {pyr};
    \node[metaboliteStyle,inputcol,left=of ptsmid] (fruc) {fruc};

    \node[metaboliteStyle,below left=of fbamid.center] (dhap) {dhap};
    \node[metaboliteStyle,below right=of fbamid] (gap) {gap};
    \node[metaboliteStyle,below=of gap.center] (bpg) {bpg};
    \node[metaboliteStyle,below=of bpg.center] (3pg) {3pg};
    \node[metaboliteStyle,below=of 3pg.center] (2pg) {2pg};
    \node[metaboliteStyle,below=of 2pg.center] (pep) {pep};
    \node[metaboliteStyle,right=of pep] (oaa) {oaa};
    \node[metaboliteStyle,below=of pep.center] (pyr) {pyr};
    \draw[figArrowStyle,line width=\frucfbp,branchoutfl] (fbp)-- (f6p) ;
    \node[anchor=west] at (fbp.east) (frucb1) {\coloredRatio{\fruccapvalfbp}{\fruccapvalfba}};
    \node[at=(frucb1.north east)] (star) {*};
    \draw [line width=\fruccapfba,autocatacyc!40] (fbp) -- (fbamid);
    \draw [capArrowStyle,line width=\fruccapfba/2,autocatacyc!40] ([xshift=-\fruccapfba/4]fbamid) [out=-90,in=45] to (dhap);
    \draw [capArrowStyle,line width=\fruccapfba/2,autocatacyc!40] ([xshift=\fruccapfba/4]fbamid) [out=-90,in=135] to (gap);
    \draw [line width=\frucfba,autocatacycfl] (fbp) [out=-90,in=90] to (fbamid);
    \draw [figArrowStyle,line width=\frucfba/2,autocatacycfl] ([xshift=-\frucfba/4]fbamid) [out=-90,in=45] to (dhap);
    \draw [figArrowStyle,line width=\frucfba/2,autocatacycfl] ([xshift=\frucfba/4]fbamid) [out=-90,in=135] to (gap);
    \draw [figArrowStyle,line width=\fructpi,autocatacycfl] (dhap) -- (gap);

    \draw[figArrowStyle,line width=\frucgap,autocatacycfl] (gap) -- (bpg);
    \draw[figArrowStyle,line width=\frucpgk,autocatacycfl] (bpg) -- (3pg);
    \draw[figArrowStyle,line width=\frucgpm,autocatacycfl] (3pg) -- (2pg);
    \draw[figArrowStyle,line width=\fruceno,autocatacycfl] (2pg) -- (pep);
    \draw[capArrowStyle=branchoutfl,line width=\fruccappyk,branchout!40] (pep) -- (pyr);
    \draw[figArrowStyle,line width=\frucpyk,branchoutfl] (pep) -- (pyr);
    \draw[capArrowStyle=branchoutfl,line width=\fruccapppc,branchout!40] (pep) -- (oaa);
    \node[ratioRect,anchor=south east] at (pep.north west) (frucb2) {\coloredRatio{\fruccapvalcomp}{\fruccapvalpts}};
    \draw[figArrowStyle,line width=\frucppc,branchoutfl] (pep) -- (oaa);
    \node[shape=coordinate,left=2.5cm of pep.center] (pts3) {};
    \draw[line width=\frucpts/2,autocatacycfl] ([yshift=\frucpts/4]pep.west) -- (pts3);
    \node[shape=coordinate,left=2.5cm of pyr.center] (pts5) {};
    \draw[figArrowStyle,line width=\frucpts/2,autocataby] ([yshift=-3/4*\frucpts]ptsmid) [out=0,in=90] to (pyr1);
    \node[shape=coordinate,xshift=-2mm] at (fbamid -| dhap.west) (ptstop) {};
    \node[shape=coordinate] at(ptstop |- 2pg.center) (ptsbottom) {};
    \draw[] (pts3) [in=-90,out=180] to (ptsbottom);
    \draw[] (ptsbottom) [in=-90,out=90] to (ptstop);
    \draw[inputcol,line width=\frucpts] (fruc) -- (ptsmid);
    \draw[line width=\frucpts/2,autocatacycfl] (pts3) [in=-90,out=180] to (ptsbottom);
    \draw[line width=\frucpts/2,autocatacycfl] (ptsbottom) -- (ptstop);
    \draw[line width=\frucpts/2,autocatacycfl] (ptstop) [in=180,out=90] to ([yshift=-3/4*\frucpts]ptsmid);
    \draw[figArrowStyle,line width=\frucpts,autocatacycfl] (ptsmid) -- (fbp);
  \end{scope}
  \end{tikzpicture}


    \end{adjustbox}
}

\frame{\frametitle{Conclusions}
\begin{itemize}
    \item Autocatalytic cycles play a major role in central carbon metabolism
    \item Proper function of autocatalytic cycles depends on kinetic parameters of enzymes
    \begin{itemize}
        \item Limits affinity of branch reactions
    \end{itemize}
    \item In metabolic engineering of autocatalytic cycles, native kinetic parameters can prohibit function
    \item Stability of autocatalytic cycles depends on under-saturation of branch reactions
    \begin{itemize}
        \item Excess expression of branch reactions enzymes is required
    \end{itemize}
   \item Fluxomics data approves sub-optimality constraints are maintained in-vivo
\end{itemize}
    \urltext
}

\frame{\frametitle{Future research questions}
\begin{itemize}
    \item What is the physical limit for lowering the activation energy barrier of a given reaction
      \pause
    \item How is the affinity of an enzyme affected by the requirement to be selective
  \end{itemize}
    \urltext
}

\frame{\frametitle{Textbook illustration}
    \begin{adjustbox}{max totalsize={0.6\textwidth}{\textheight},center}
        \pgfimage{energyLandscapeLeninger.png}
    \end{adjustbox}
    \urltext
}

\frame{\frametitle{Modeling energy landscape modification in a classical system}
\only<1>{
    \begin{adjustbox}{max totalsize={0.6\textwidth}{\textheight},center}
      \pgfimage{subsbody2.pdf}
    \end{adjustbox}
  }
\only<2>{
    \begin{adjustbox}{max totalsize={0.6\textwidth}{\textheight},center}
      \pgfimage{subsbody3.pdf}
    \end{adjustbox}
  }
 \only<3>{
    \begin{adjustbox}{max totalsize={0.6\textwidth}{\textheight},center}
      \pgfimage{subsbody4.pdf}
    \end{adjustbox}
  }
 \only<4>{
    \begin{adjustbox}{max totalsize={0.6\textwidth}{\textheight},center}
      \pgfimage{subsbody5.pdf}
    \end{adjustbox}
  }
    \urltext
}

\frame{\frametitle{Reaction energy landscape of model substrate}
\begin{columns}
    \begin{column}{0.5\textwidth}
        \pgfimage[width=\textwidth]{subsbody5.pdf}
    \end{column}
    \begin{column}{0.5\textwidth}
        \pgfimage[width=\textwidth]{energylandscapemodificationmodel.pdf}
    \end{column}
\end{columns}
    \urltext
}

\frame{\frametitle{Introducing a model catalyst}
    \begin{adjustbox}{max totalsize={0.6\textwidth}{\textheight},center}
      \only<1>{
        \pgfimage[width=\textwidth]{subscatdist.pdf}
      }
      \only<2>{
        \pgfimage[width=\textwidth]{subscatdistforce.pdf}
      }
     \end{adjustbox}
    \urltext
}

\frame{\frametitle{Reaction energy landscape of bound substrate}
\begin{columns}
    \begin{column}{0.5\textwidth}
        \pgfimage[width=\textwidth]{subscatforces.pdf}
     \end{column}
    \only<2>{
    \begin{column}{0.5\textwidth}
        \pgfimage[width=\textwidth]{energylandscapemodificationmodel2.pdf}
      \end{column}
    }
\end{columns}
    \urltext
}

\frame{\frametitle{The catalyst creates a bypass to the energy barrier at the transition state}
    \begin{adjustbox}{max totalsize={0.6\textwidth}{\textheight},center}
        \pgfimage{catlandscape.pdf}
    \end{adjustbox}
    \urltext
}

\frame{\frametitle{Subtracting the potential field at the transition state from the initial state produces an energy barrier reduction landscape}
    \only<1-2>{
\begin{columns}
    \begin{column}{0.5\textwidth}
      \pgfimage[width=\textwidth]{subsstartpot.pdf}
    \end{column}
    \begin{column}{0.5\textwidth}
    \only<2>{
        \pgfimage[width=\textwidth]{substranspot.pdf}
    }
    \end{column}
\end{columns}
}
    \only<3>{
    \begin{adjustbox}{max totalsize={0.6\textwidth}{\textheight},center}
        \pgfimage{subsdiffpot.pdf}
    \end{adjustbox}
     }
    \urltext
}

\frame{\frametitle{Subtracting the potential field at the transition state from the initial state produces an energy barrier reduction landscape}
\begin{itemize}
    \item The resulting function quantifies the barrier reduction when positioning a positive point charge at any coordinate in space
      \pause
    \item Placing charges at extremum points of this function achieves maximal barrier reduction
  \end{itemize}
    \urltext
}


\frame{\frametitle{Methodological approach for investigating catalytic constraints}
\begin{itemize}
      \pause
     \item Crowd-sourcing platform
      \begin{itemize}
           \item Challenge existing assumptions
           \item Reveal potential catalytic mechanisms
      \end{itemize}
      \pause
     \item Handheld, functional models
      \begin{itemize}
       \item Demonstrate and communicate catalytic principles
      \end{itemize}
      \pause
     \item Apply theoretical framework to molecular domain
      \pause
     \item Investigate metabolic network design implications
      \begin{itemize}
       \item Synthetic biology applications
       \item Origins of life metabolism
      \end{itemize}
  \end{itemize}
    \urltext
}


\frame{\frametitle{Acknowledgments}
\begin{columns}
    \begin{column}{0.7\textwidth}
        \pgfimage[width=\textwidth]{lab_2016.png}
    \end{column}
    \begin{column}{0.3\textwidth}
        \pgfimage[width=0.5\textwidth]{eladnoor.png}

        \pgfimage[width=0.5\textwidth]{ed.jpg}

        \pgfimage[width=0.9\textwidth]{aeri.png}

        \pgfimage[width=0.9\textwidth]{erc.jpg}
    \end{column}
\end{columns}
}

\frame{\frametitle{Summary}
\begin{itemize}
  \item Basic challenges of biological systems are rarely investigated theoretically
     \item Transforming key problems to simplified models in accessible platforms can leverage innovation of wider audience and reveal novel principles
     \item Recently available datasets allow evaluation of hypotheses
     \item Mapping metabolic networks into the chemical space can highlight metabolic network motifs

\end{itemize}
References (autocatalysis):
\begin{itemize}
    \item Carbon fixation in \emph{E.coli}: Antonovsky et. al., Cell 2016
    \item Emergence of autocatalysis in metabolic networks: Riehl et. al., PLoS CB 2010
    \item Algorithms for identifying autocatalytic cycles: Kun et. al., Genome Biology 2008
    \item Calculating $k_\text{cat}$ from proteomics data: Davidi et. al., PNAS 2016
    \item This work: Barenholz et. al., eLife 2017

\end{itemize}
Thank you!
    \urltext
}

\backupbegin

\frame{\frametitle{Supplementary figures and data}
}

\frame{\frametitle{Outlook}
\begin{itemize}
    \item Efficient algorithm for identification of autocatalytic cycles in large metabolic networks
    \item Experimental exploration of different autocatalytic cycles function in-vivo
    \item Possible other uses of passive control of metabolic fluxes due to kinetic parameters
\end{itemize}
References:
\begin{itemize}
    \item Carbon fixation in \emph{E.coli}: Antonovsky et. al., Cell 2016
    \item Emergence of autocatalysis in metabolic networks: Riehl et. al., PLoS CB 2010
    \item Algorithms for identifying autocatalytic cycles: Kun et. al., Genome Biology 2008
    \item Calculating $k_\text{cat}$ from proteomics data: Davidi et. al., PNAS 2016
    \item This work: Barenholz et. al., eLife 2017

\end{itemize}
    \urltext
}

\frame{\frametitle{Does structural similarity limit affinity in metabolic networks?}
\pause
\vfill
\vfill
    \begin{adjustbox}{max totalsize={0.6\textwidth}{\textheight},center}
        \pgfimage{babyToy.jpg}
    \end{adjustbox}
    \urltext
}

\frame{\frametitle{Does structural similarity limit affinity in metabolic networks?}
\begin{itemize}
  \item Most enzymes are substrate-specific
    \pause
   \item Structural similarity is used for drug discovery and promiscuous activity tests
     \pause
   \item Metabolic networks must contain structurally similar metabolites
    \begin{itemize}
         \item But can potentially reduce similarities at critical points
     \end{itemize}
     \pause
     \item Numerous examples for specificity tradeoffs in the literature
  \end{itemize}
    \urltext
}

\frame{\frametitle{Why do we expect selectivity to decrease affinity?}
    \only<1>{
    \begin{adjustbox}{max totalsize={0.6\textwidth}{\textheight},center}
        \pgfimage{twoshapes.pdf}
      \end{adjustbox}}
      \only<2>{
    \begin{adjustbox}{max totalsize={0.6\textwidth}{\textheight},center}
        \pgfimage{twoshapesslack.pdf}
      \end{adjustbox}}
     \urltext
}


\frame{\frametitle{Examples of specificity-affinity challenges}
\begin{columns}
    \begin{column}{0.5\textwidth}
 \begin{itemize}
    \only<1->{
     \item RuBisCo
      \begin{itemize}
        \item \ce{CO2} versus \ce{O2}
      \end{itemize}
    }
    \only<2->{
     \item Tyrosine ammonia lyase
      \begin{itemize}
        \item \ce{Tyr} versus \ce{Phe}
      \end{itemize}
    }
    \only<3->{
    \item Bacterial DNA methyltransferase
      \begin{itemize}
        \item Relaxing sequence specificity accelerates rate
      \end{itemize}
    \item Bacterial hexose phosphate transporter
    }
       
  \end{itemize}
    \end{column}
    \begin{column}{0.5\textwidth}
    \only<1>{
    \begin{adjustbox}{max totalsize={0.6\textwidth}{\textheight},center}
        \pgfimage{co2o2.jpg}
      \end{adjustbox}}
    \only<2>{
    \begin{adjustbox}{max totalsize={0.6\textwidth}{\textheight},center}
        \pgfimage{tyrosinePhenylalenine.jpg}
      \end{adjustbox}}
      \end{column}
\end{columns}
    \urltext
}

\frame{\frametitle{Can we formulate a quantitative evaluation of the selectivity challenge?}
\begin{itemize}
     \item Given metabolites concentration data
      \begin{itemize}
       \item Identify challenging reactions
       \item Quantify expected cost
      \end{itemize}
      \pause
     \item Given reaction possibilities
      \begin{itemize}
       \item Find biases in metabolic network structure maximizing structural differences
      \end{itemize}
  \end{itemize}
    \urltext
}

\frame{\frametitle{Methodological approach for investigating selectivity tradeoffs}
\begin{itemize}
     \item Impact on metabolites concentrations and enzymes
   \begin{itemize}
       \item BRENDA - identifying weak affinity enzymes
       \item Promiscuous activity data from Sauer lab
       \item Structural similarity metrics comparison with measured metabolites concentrations
   \end{itemize}
   \pause
 \item Impact on network structure
   \begin{itemize}
     \item Project metabolic networks to chemical space
     \item Implement selectivity in constraint based modeling of metabolic networks
   \end{itemize}
 \end{itemize}
    \urltext
}
 
\frame{\frametitle{Fructose PTS disagreement results from missing data on alternative transport pathways}
\begin{itemize}
    \item All fructose was assumed to be transported as fbp
    \item Experimental evidence shows other transport pathways are functioning\footnote{Kornberg, 1990}
\end{itemize}
}

\frame{\frametitle{Allosteric regulation can accelerate convergence to steady state and increase robustness in fluctuating environment}
\begin{itemize}
    \item Convergence to steady state is faster when the differences between the cycle flux and the branch flux are larger
    \begin{itemize}
        \item Intermediate metabolites activate branch reactions and inhibit cycle reactions
    \end{itemize}
    \pause
    \item Adaptation of steady state fluxes to nutrient availability is achieved by allosteric regulation of the assimilated metabolite
    \begin{itemize}
        \item The assimilated metabolite should activate branch reactions and inhibit cycle reactions
    \end{itemize}
    \pause
    \item Adaptation of steady state fluxes to demand of cycle products is achieved by allosteric regulation of the branch products
    \begin{itemize}
        \item Branch products should activate branch reactions and inhibit cycle reactions
    \end{itemize}
    \pause
    \item For the PTS using cycle, 11 out of 12 allosteric interactions agree with these predictions

\end{itemize}
    \urltext
}

\frame{\frametitle{Input flux increases the range of parameters for which stable fluxes exist}
    \begin{adjustbox}{max totalsize={\textwidth}{0.8\textheight},center}
        \begin{tikzpicture}[>=latex',node distance = 2cm]
  \begin{scope}[]
        \node at (-60:1cm) (X) {$X$};
        \node[shape=coordinate] (orig) {};
        \draw [-,line width=1pt,autocatacyc] (X.south west) arc (285:10:1cm) node [pos=0.65,above] (fa) {$f_a:$\small{$A+X\rightarrow2X$}} node [pos=0.45,shape=coordinate] (midauto) {} node [pos=1,shape=coordinate] (endcommon) {};
        \draw [->,line width=1pt,autocatacyc] (endcommon) arc (-17:-46:1.5cm);
        \draw [->,line width=1pt,autocatacyc] (endcommon) arc (3:-35:1.2cm);
        \draw [line width=1pt,assimcol] (midauto) arc (-60:-90:1cm) node [pos=1,left] (e) {$A$};
        \draw [->,line width=1pt,branchout] (X.south east) arc (225:270:1cm) node [pos=0.75,above] {$f_b$};
    \draw [<-,line width=1pt,magenta] (X.east) arc (-90:-30:1cm) node [pos=0.4,above] (fi) {$f_i$};
        \iftoggle{article} {
            \node at (-2.4cm,1.3cm) (A) {(A)};
        }{}
  \end{scope}
    \begin{customlegend}[legend entries={$f_a+f_i$,$f_b$,$\dot{X}=f_a+f_i-f_b$},legend style={right=3cm of orig,anchor=west,name=legend1}]
      \addlegendimage{autocatacyc,fill=black!50!red,sharp plot,line width=1pt}
      \addlegendimage{branchout,fill=black!50!red,sharp plot,line width=1pt}
      \addlegendimage{sumcolor,fill=black!50!red,sharp plot,line width=1pt}
    \end{customlegend}
    \begin{scope}[shift={(-2.5cm,-6cm)},node distance = 1cm]

        \begin{axis}[name=plot1,axis x line=middle,axis y line=left,xlabel near ticks,ylabel near ticks,xmin=0,ymin=-1.5,xmax=2.9,ymax=4.8,xlabel={[$X$]},ylabel={flux},samples=60,width=5cm,height=6cm,yticklabels={,,},xticklabels={,,},tick label style={major tick length=0pt}]
    \addplot[domain=0:4,autocatacyc,thick] {3*x/(0.5+x)+1.5*\influx};
    \addplot[domain=0:4,branchout,thick] {5*x/(1+x)};
    \addplot[domain=0:4,sumcolor,thick] {3*x/(0.5+x)-5*x/(1+x)+1.5*\influx};
    \addplot[dashed,gray,thick] coordinates {(1.5,0) (1.5,3)};
    \node[right] (one) at (axis cs:0.1,4.5) {I};
    \node[right,align=left] (onetext) at (axis cs:0.05,3.7) {\scriptsize stable non-zero\\[-0.4em]\scriptsize steady state};
  \end{axis}

  \begin{axis}[name=plot2,axis x line=middle,axis y line=left,xlabel near ticks,ylabel near ticks,xmin=0,ymin=-1.5,xmax=2.9,ymax=4.8,xlabel={[$X$]},ylabel={flux},samples=60,at=(plot1.right of south east),anchor=left of south west,width=5cm,height=6cm,yticklabels={,,},xticklabels={,,},tick label style={major tick length=0pt}]
    \addplot[domain=0:4,autocatacyc,thick] {4.7*x/(1+x)+1.5*\influx};
    \addplot[domain=0:4,branchout,thick] {4*x/(1+x)};
    \addplot[domain=0:4,sumcolor,thick] {5*x/(1+x)-4*x/(1+x)+1.5*\influx};
    \node[right] (two) at (axis cs:0.1,4.5) {II};
    \node[right,align=left] (twotext) at (axis cs:0.05,4) {\scriptsize no steady state};
  \end{axis}

  \begin{axis}[name=plot3,axis x line=middle,axis y line=left,xlabel near ticks,ylabel near ticks,xmin=0,ymin=-1.5,xmax=2.9,ymax=4.8,xlabel={[$X$]},ylabel={flux},samples=60,at=(plot2.right of south east),anchor=left of south west,width=5cm,height=6cm,yticklabels={,,},xticklabels={,,},tick label style={major tick length=0pt}]
    \addplot[domain=0:4,autocatacyc,thick] {6*x/(2+x)+1.5*\influx};
    \addplot[domain=0:4,branchout,thick] {4*x/(0.4+x)};
    \addplot[domain=0:4,sumcolor,thick] {6*x/(2+x)-4*x/(0.4+x)+1.5*\influx};
    \addplot[dashed,gray,thick] coordinates {(1.2,0) (1.2,3)};
    \addplot[dashed,gray,thick] coordinates {(0.182,0) (0.182,1.25)};
    \node[right] (three) at (axis cs:0.1,4.5) {III};
    \node[right,align=left] (threetext) at (axis cs:0.05,3.7) {\scriptsize two non-zero\\[-0.4em]\scriptsize steady states};
  \end{axis}
    \end{scope}
\end{tikzpicture}


    \end{adjustbox}
}

\frame{\frametitle{Additional autocatalytic cycles in central carbon metabolism}
    \begin{adjustbox}{max totalsize={\textwidth}{0.8\textheight},center}
        \begin{tikzpicture}
  \colorlet{ppinit}{magenta}

  \pgfmathsetlength{\assimwidth}{1.5pt};

  \begin{scope}

  \node[metaboliteStyle] (g6p) {g6p};


  \node[metaboliteStyle,below=of g6p.center] (f6p) {f6p};
  \node[metaboliteStyle,below=of f6p] (fbp) {fbp};
  \node[metaboliteStyle,shape=coordinate,below=of fbp.center](fbamid) {};
  \node[metaboliteStyle,below left=of fbamid.center] (dhap) {dhap};
  \node[metaboliteStyle]at([xshift=1.4cm]dhap -|fbamid.center) (gap) {gap};
  \node[metaboliteStyle,right=of g6p] (6pgi) {6pgi};
  \node[metaboliteStyle,shape=coordinate,right=of f6p] (s7pspace) {};
  \node[metaboliteStyle,right=of s7pspace] (s7p) {s7p};
  \node[metaboliteStyle,] at (fbp.center -| s7p.center) (e4p) {e4p};
  \node[metaboliteStyle,right=of e4p] (xu5p) {xu5p};
  \node[metaboliteStyle,above right=of xu5p] (ru5p) {ru5p};
  \node[metaboliteStyle,right=of ru5p,yshift=0.2cm,xshift=-0.2cm] (co2) {\ce{CO2}};
  \node[metaboliteStyle,above=of ru5p.center] (6pgc) {6pgc};
  \node[metaboliteStyle,above=of xu5p,yshift=2.5cm,rectangle,draw=assimcol,rounded corners=2pt] (r5p) {r5p};
  \node[shape=coordinate] at ([shift={(-0.8cm,-5mm)}]s7p -|r5p) (midtkt1) {};
  \path[] (e4p) -- (gap) coordinate [pos=0.4] (midtkt2) {};
  \path[] (e4p) -- (s7pspace) coordinate [pos=0.5] (midtal) {};
  \draw[<-] (g6p) -- (f6p);
  \draw[<-] (f6p.south) -- (fbp.north);
  \draw [] (fbamid) [out=-90,in=45] to (dhap);
  \draw [] (fbamid) [out=-90,in=135] to (gap);
  \draw[->] (ru5p) -- (xu5p);
  \draw[-] (xu5p) [out=180,in=-90] to (midtkt1);
  \draw[->] (midtkt1) [out=90,in=0] to (s7p);
  \draw[assimcol,line width=\assimwidth] (r5p) [out=-110,in=90] to (midtkt1);
  \draw[->] (midtkt1) [out=270,in=0] to ([yshift=1mm]gap.east);
  \draw[] (e4p) [out=-60,in=0] to (midtkt2);
  \draw[->] (midtkt2) [out=180,in=90] to (gap);
  \draw[] (xu5p) [out=245,in=0] to (midtkt2);
  \draw[->] (midtkt2) [out=180,in=-30] to (f6p);
  \draw[->] (midtal) [out=90,in=0] to (f6p);
  \draw[] (s7p) [out=180,in=90] to (midtal);
  \draw[->] (midtal) [out=-90,in=180] to (e4p);
  \draw[] (gap) [out=55,in=-90] to (midtal);
  \draw [<-] (fbp) [out=-90,in=90] to (fbamid);
  \draw [<-] (dhap) -- (gap);
  \draw[->] (g6p) -- (6pgi);
  \draw[->] (6pgi) -- (6pgc);
  \draw[->] (6pgc) -- (ru5p);
  \draw[->] ([yshift=-1mm] 6pgc) [out=-90,in=180] to (co2);
  \draw[<-,dashed] (ru5p) [out=180,in=-45] to (r5p);
  
  \draw[opacity=0.2,fill=ppinit,rounded corners=\highlightrad,even odd rule] ([shift={(\highlightrad,\highlightrad)}] 6pgc.north east) -- ([shift={(\highlightrad,-\highlightrad)}]ru5p.south -| 6pgc.east) -- ([xshift=5mm]dhap.south  -| gap.east) -- node [midway] (ppshade) {} (dhap.south west) -- (dhap.north west) -- ([xshift=-\highlightrad]fbamid  -| g6p.west) -- ([shift={(-\highlightrad,\highlightrad)}] g6p.north west)--cycle;

  %\draw[very thick,dashed,ppinit,->,visible on=<4->] (ppshade) -- ++(0cm,-1.1cm); 
  \end{scope}

  %% r5p cycle
  \begin{scope} [shift={(-5cm,-2cm)},radius=2cm]
  \draw[lightgray,rounded corners=\highlightrad] (-2.4cm,-2.1cm) rectangle +(5.1,4.6);

  \node[anchor=north] at(0.3cm,-2.2cm) (glyreac) {{\fontfamily{cmss}\selectfont 4} xu5p + {\fontfamily{cmss}\selectfont 2} r5p $\rightarrow$ {\fontfamily{cmss}\selectfont 5} xu5p + {\fontfamily{cmss}\selectfont 5} \ce{CO2}};
    \pgfmathsetlength{\ptsierad}{\autocatalrad*0.5};
    \pgfmathsetlength{\ptsimrad}{\autocatalrad-0.5*\arcwidth};

    \assim{2*\arcwidth}{100}{-30}{\ptsimrad}{3/2}
    \arrowhead{2*\arcwidth}{-45}{\autocatalrad}{ppinit}

    \pgfmathsetlength{\ptsesrad}{\autocatalrad+1/4*\arcwidth};
    \pgfmathsetmacro{\startbranch}{-15}
    \colorgradarc{5/2*\arcwidth}{40}{\startbranch}{\ptsesrad}{autocatacyc}{ppinit}


    \draw[color=autocatacyc,line width=3*\arcwidth] (100:\autocatalrad) arc(100:40:\autocatalrad);

    \draw[color=ppinit,line width=2*\arcwidth] (\startbranch:\autocatalrad) arc(\startbranch:-45:\autocatalrad);

    \pgfmathsetlength{\ptsarcwidth}{\autocatalrad-\arcwidth/2};

    \shadedarc[2*\arcwidth]{-95}{-260}{\autocatalrad}{\ptsarcwidth}{autocatacyc}{ppinit}


    \shadedarc[\arcwidth/2]{40}{-15}{\ptsarcwidth-3/4*\arcwidth}{\ptsarcwidth-3/4*\arcwidth}{white}{autocatacyc}

    \begin{scope}[shift={(\startbranch:2*\autocatalrad+5/4*\arcwidth)}]
        \draw[color=ppinit,line width=\arcwidth/2] (\startbranch+180:\autocatalrad) arc (\startbranch+180:\startbranch+180+25:\autocatalrad);
        \revarrowhead{\arcwidth/2}{190}{\autocatalrad}{ppinit}
    \end{scope}

    \node at (-37:\autocatalrad+3.5*\arcwidth) (xu5p) {+xu5p};

    \node at(135:\autocatalrad+2.5*\arcwidth) (r5p) {{\fontfamily{cmss}\selectfont 2} r5p};

    \node at (-70:\autocatalrad+\arcwidth/3) (xu5p) {{\fontfamily{cmss}\selectfont 4} xu5p};
    \node[gray] at (-35:2*\arcwidth) (co2) {{\fontfamily{cmss}\selectfont 5} \ce{CO2}};
  \end{scope}
\end{tikzpicture}


    \end{adjustbox}
}

\frame{\frametitle{Additional autocatalytic cycles in central carbon metabolism}
    \begin{adjustbox}{max totalsize={\textwidth}{0.8\textheight},center}
        \begin{tikzpicture}
  \colorlet{fbainit}{cyan}

  \pgfmathsetlength{\assimwidth}{1.5pt};

  \begin{scope}
  \node[metaboliteStyle] (g6p) {g6p};


  \node[metaboliteStyle,below=of g6p.center] (f6p) {f6p};
  \node[metaboliteStyle,below=of f6p] (fbp) {fbp};
  \node[metaboliteStyle,shape=coordinate,below=of fbp.center](fbamid) {};
  \node[metaboliteStyle,below left=of fbamid.center,rectangle,draw=assimcol,rounded corners=2pt] (dhap) {dhap};
  \node[metaboliteStyle]at([xshift=1.4cm]dhap -|fbamid.center) (gap) {gap};
  \node[metaboliteStyle,below=of gap.center] (bpg) {bpg};
  \node[metaboliteStyle,below=of bpg.center] (3pg) {3pg};
  \node[metaboliteStyle,below=of 3pg.center] (2pg) {2pg};
  \node[metaboliteStyle,below=of 2pg.center] (pep) {pep};
  \node[metaboliteStyle,below=of pep.center] (pyr) {pyr};
  \node[metaboliteStyle,right=of g6p] (6pgi) {6pgi};
  \node[metaboliteStyle,right=of 6pgi] (6pgc) {6pgc};
  \draw[<-] (g6p) -- (f6p);
  \draw[<-] (f6p.south) -- (fbp.north);
  \draw [assimcol,line width=\assimwidth] (fbamid) [out=-90,in=45] to (dhap);
  \draw [] (fbamid) [out=-90,in=135] to (gap);
  \draw[<-] (3pg) -- (2pg);
  \draw[<-] (2pg) -- (pep);
  \draw[<-] (pep) -- (pyr);
  \draw[<-] (gap) -- (bpg);
  \draw[<-] (bpg) -- (3pg);
  \draw [<-] (fbp) [out=-90,in=90] to (fbamid);
  \draw [dashed,->] (dhap) -- (gap);
  \draw[->] (g6p) -- (6pgi);
  \draw[->] (6pgi) -- (6pgc);
  \node[metaboliteStyle,below=of 6pgc] (kdg) {kdg};
  \node[shape=coordinate,] at (fbamid -| kdg.center) (eddtop) {};
  \node[shape=coordinate,] at (2pg.center -| kdg.center) (eddbottom) {};
  \draw[->] (6pgc) -- (kdg);
  \draw[] (kdg) [out=-90,in=90] to (eddtop);
  \draw[->] (eddtop) [out=-90,in=0] to (gap);
  \draw[] (eddtop) [out=-90,in=90] to (eddbottom);
  \draw[->] (eddbottom) [out=-90,in=0] to (pyr);
  
  \draw[opacity=0.2,fill=fbainit,rounded corners=\highlightrad] ([shift={(-\highlightrad,\highlightrad)}]g6p.north west) -- ([shift={(\highlightrad,\highlightrad)}] 6pgc.north east) -- ([shift={(\highlightrad,\highlightrad)}]kdg.north east)-- ([shift={(\highlightrad,-\highlightrad)}]2pg.south -| kdg.east) -- ([shift={(\highlightrad,-\highlightrad)}]pyr.south -| kdg.east) -- ([shift={(-\highlightrad,-\highlightrad)}] pyr.south west) -- node [midway] (fbashade) {} ([xshift=-\highlightrad] gap.west) -- ([shift={(-\highlightrad,-\highlightrad)}] fbamid.south -| g6p.west) -- cycle;

%  \draw[very thick,dashed,fbainit,->,visible on=<4->] (fbashade) -- ++(-1.6cm,0cm); 
  \end{scope}

  %% FBA cycle
  \begin{scope} [shift={(-3.5cm,-9cm)},radius=2cm]
  \draw[lightgray,rounded corners=\highlightrad] (-2.4cm,-2.1cm) rectangle +(5.1,4.6);

  \node[anchor=north] at(0cm,-2.2cm) (glyreac) {gap + dhap $\rightarrow$ {\fontfamily{cmss}\selectfont 2} gap};

    \preassim{\arcwidth}{-90}{-270}{\autocatalrad}{fbainit}
    \postassim{\arcwidth}{90}{-45}{\autocatalrad}{fbainit}{2}
    \assim{\arcwidth}{90}{-30}{\autocatalrad}{2}
    \arrowhead{\arcwidth}{-45}{\autocatalrad}{fbainit}

    \node at (-37:\autocatalrad+3.5*\arcwidth) (gapp) {+gap};

    \node at(127:\autocatalrad+2.5*\arcwidth) (dhap) {dhap};

    \node at (-70:\autocatalrad) (gap) {gap};
  \end{scope}

\end{tikzpicture}


    \end{adjustbox}
}

\frame{\frametitle{ATP autocatalysis in glycolysis}
    \begin{adjustbox}{max totalsize={\textwidth}{0.8\textheight},center}
        \newlength\cyclerad
\pgfmathsetlength{\cyclerad}{2cm}
\begin{tikzpicture}
  \node[] at (0:\cyclerad) (atp) {atp};
  \node[gray] at (350:1.5*\cyclerad) (gluc) {gluc};
  \node[gray] at (330:1.5*\cyclerad) (adp1) {adp};
  \node[] at (320:\cyclerad) (g6p) {g6p};
  \node[] at (280:\cyclerad) (f6p) {f6p};
  \node[] at (235:\cyclerad) (fbp) {fbp};
  \node[gray] at (250:1.5*\cyclerad) (adp2) {adp};
  \node[] at (200:1.5*\cyclerad) (dhap) {dhap};
  \node[] at (180:\cyclerad) (gap) {gap};
  \node[gray] at (170:1.5*\cyclerad) (pi) {p};
  \node[] at (135:\cyclerad) (bpg) {bpg};
  \node[gray] at (120:1.5*\cyclerad) (adp3) {adp};
  \node[] at (100:\cyclerad) (3pg) {3pg};
  \node[] at (65:\cyclerad) (2pg) {2pg};
  \node[] at (30:\cyclerad) (pep) {pep};
  \node[gray] at (25:1.5*\cyclerad) (adp4) {adp};
  \node[gray] at (7:1.5*\cyclerad) (pyr) {pyr};
  \draw[->] (atp) [out=-135,in=0] to node [pos=0.9,shape=coordinate] (midpfk) {} (fbp);
  \draw[] (f6p) [out=140,in=0] to (midpfk);
  \draw[->] (midpfk) [out=190,in=90] to (adp2);
  \draw[->] (g6p) [out=220,in=20] to (f6p);
  \draw[->] (atp) [out=-90,in=60] to node [pos=0.5,shape=coordinate] (midpts) {} (g6p);
  \draw[] (gluc) [out=180,in=60] to (midpts);
  \draw[->] (midpts) [out=240] to (adp1);
  \draw[->] (fbp) [out=140,in=-85]to node [pos=0.3,shape=coordinate] (midfba) {} (gap) ;
  \draw[->] (midfba) [out=135,in=0] to (dhap);
  \draw[->] (dhap) -- (gap);
  \draw[->] (gap) [out=85,in=235] to node [pos=0.5,shape=coordinate] (mid3pg) {} (bpg) ;
  \draw[] (pi) [out=0,in=240] to (mid3pg);
  \draw[->] (bpg) [out=0,in=150] to node [pos=0.07,shape=coordinate] (midbpg) {} (atp);
  \draw[] (adp3) [out=-80,in=170] to (midbpg);
  \draw[->] (midbpg) [out=10,in=250] to (3pg);
  \draw[->] (3pg)  [out=0,in=166] to (2pg);
  \draw[->] (2pg) [out=-38,in=135] to (pep);
  \draw[->] (pep) [out=-59,in=90] to node [pos=0.5,shape=coordinate] (midpck) {}(atp);
  \draw[] (adp4) [out=210,in=120] to (midpck);
  \draw[->] (midpck) [out=300,in=180] to (pyr);
 \end{tikzpicture}

    \end{adjustbox}
}

\frame{\frametitle{Supplementary equations}
}

\frame{\frametitle{Bisubstrate reaction equations}
\begin{itemize} 
    \item Substituted enzyme
        \begin{equation*}
              f=\frac{V_{\max}AX}{K_XA+K_{A}X+AX}
        \end{equation*}
    \item Random binding ternary complex
        \begin{equation*}
          f=\frac{V_{\max}AX}{K_{i,A}K_X+K_XA+K_AX+AX}
        \end{equation*}
    \item Ordered binding ternary complex, assimilated metabolite binding first
        \begin{equation*}
            f=\frac{V_{\max}AX}{K_{i,A}K_X+K_XA+AX}
        \end{equation*}
    \item Ordered binding ternary complex, internal metabolite binding first
        \begin{equation*}
            f=\frac{V_{\max}AX}{K_{i,X}K_A+K_AX+AX}
        \end{equation*}
\end{itemize}
}

\frame{\frametitle{Reversible reaction equation}
   \begin{equation*}
       f_b=\frac{V_{\max,b}(X-Y)}{K_X+X+\frac{K_X}{K_Y}Y}
   \end{equation*}
}

\frame{\frametitle{Work plan}
    \begin{adjustbox}{max totalsize={\textwidth}{\textheight},center}
        \pgfimage{gantt.pdf}
    \end{adjustbox}
    \urltext
}

\frame{\frametitle{Catalyst design must track the entire reaction pathway}
    \begin{adjustbox}{max totalsize={0.6\textwidth}{\textheight},center}
        \pgfimage{energylandscapemodificationmodelMax.pdf}
    \end{adjustbox}
    \urltext
}

\frame{\frametitle{References}
\scriptsize{
\begin{enumerate}
  \item Cooper S, et al. (2010) Predicting protein structures with a multiplayer online game. Nature
  \item Cao Y, et al. (2008) ChemmineR: a compound mining framework for R. Bioinformatics 
  \item Reymond J-L (2015) The chemical space project. Acc Chem Res
  \item Wang Y, et al. (2013) fmcsR: mismatch tolerant maximum common substructure searching in R. Bioinformatics
  \item Bar-Even A, et al. (2015) The Moderately Efficient Enzyme: Futile Encounters and Enzyme Floppiness. Biochemistry
  \item Alam MT, et al. (2017) The self-inhibitory nature of metabolic networks and its alleviation through compartmentalization. Nat Commun
  \item Schomburg I, et al. (2004) BRENDA, the enzyme database: updates and major new developments. Nucleic Acids Res
  \item Sévin DC, et al. (2017) Nontargeted in vitro metabolomics for high-throughput identification of novel enzymes in Escherichia coli. Nat Methods
  \item Savir Y, et al. (2010) Cross-species analysis traces adaptation of Rubisco toward optimality in a low-dimensional landscape. PNAS
  \item Tcherkez GGB, et al. (2006) Despite slow catalysis and confused substrate specificity, all ribulose bisphosphate carboxylases may be nearly perfectly optimized. PNAS
  \item Danos V, et al. (2015) Rigid Geometric Constraints for Kappa Models. Electron Notes Theor Comput Sci
\end{enumerate}}
    \urltext
}

\backupend

\end{document}
