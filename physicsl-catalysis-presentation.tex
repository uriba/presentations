\documentclass[aspectratio=169]{beamer}
\usepackage{color,amsmath,amssymb,graphicx,subcaption,geometry,mathtools,xfrac}
%\usepackage{cite}
\usepackage{mhchem}
\usepackage{tikz}
\usepackage{pgfplots}
\usepackage{lineno}
\pgfplotsset{compat=1.12}
\usepackage{stackengine,ifthen}
\usepackage{float}
\usetikzlibrary{arrows,positioning,calc,arrows.meta,patterns,fit}

\newtoggle{article}
\newtoggle{poster}
\newtoggle{eddpathway}
\newtoggle{elifesubmission}
\newtoggle{thesis}
\togglefalse{elifesubmission}
%\toggletrue{elifesubmission}

\iftoggle{elifesubmission} {
\usepackage{setspace}
\linenumbers
\doublespacing
\newcommand{\beginsupplement}{%
        \setcounter{table}{0}
        \renewcommand{\thetable}{S\arabic{table}}%
        \setcounter{figure}{0}
        \renewcommand{\thefigure}{2-figure supplement \arabic{figure}}%
    }
}
{
\newcommand{\beginsupplement}{%
        \setcounter{table}{0}
        \renewcommand{\thetable}{S\arabic{table}}%
        \setcounter{figure}{0}
        \renewcommand{\thefigure}{S\arabic{figure}}%
    }
}


\tikzset{>=Latex}
\newcommand\influx{0.5}

\newenvironment{customlegend}[1][]{
  \begingroup
  \csname pgfplots@init@cleared@structures\endcsname
  \pgfplotsset{#1}
}{
  \csname pgfplots@createlegend\endcsname
  \endgroup
}

\def\addlegendimage{\csname pgfplots@addlegendimage\endcsname}
\setlength\abovecaptionskip{6pt}
\providecommand{\abs}[1]{\lvert#1\rvert}
\providecommand{\norm}[1]{\lVert#1\rVert}

\tikzset{>=latex}
\tikzset{metaboliteStyle/.style={}}
\definecolor{cyan}{RGB}{100,181,205}
\definecolor{blue}{RGB}{76,114,176}
\definecolor{green}{RGB}{85,168,104}
\definecolor{magenta}{RGB}{129,114,178}
\definecolor{yellow}{RGB}{204,185,116}
\definecolor{red}{RGB}{196,78,82}
\definecolor{graybg}{gray}{0.95}

\colorlet{assimcol}{green}
\colorlet{sumcolor}{yellow}

\colorlet{inputcol}{green}
\colorlet{branchout}{red}
\colorlet{branchoutfl}{red!80}
\colorlet{autocatacyc}{blue}
\colorlet{autocatacycfl}{blue!80}
\colorlet{autocataby}{cyan}

\pdfpageattr{/Group <</S /Transparency /I true /CS /DeviceRGB>>} 

\def\blendfrac{0.5}
\def\deltaang{-155}
\def\fromang{180}
\def\inputang{-40}
\def\protrude{7}
\def\arcwidth{0.3cm}
\def\highlightrad{0.2cm}
\def\autocatalrad{1.5cm}
\def\autocatalscale{1.5}

  \newcommand{\colorgradarc}[6]{%width,startang,stopang,rad,startcol,stopcol
    \pgfmathsetmacro\arcrange{#3-#2}
    \pgfmathsetmacro\progsign{\arcrange>0 ? 1 : -1}
    \pgfmathsetmacro\arcend{#3-1}
    \foreach \i in {#2,...,\arcend} {
     \pgfmathsetmacro\fracprog{\i/\arcrange-#2/\arcrange}
     \pgfmathsetmacro\col{\fracprog*100}
     \draw[color={#6!\col!#5},line width=#1] (\i-\progsign:#4)
         arc[start angle=\i-\progsign, end angle=\i+1.1*\progsign,radius=#4];
    }
  }

  \newcommand{\shadedarc}[7][\arcwidth]{%width,startang,stopang,startrad,stoprad,startcol,stopcol
    \pgfmathsetmacro\arcrange{#3-#2}
    \pgfmathsetmacro\radrange{#5-#4}
    \pgfmathsetmacro\progsign{\arcrange>0 ? 1 : -1}
    \foreach \i in {#2,...,\numexpr#3-1\relax} {
      \pgfmathsetmacro\fracprog{\i/\arcrange-#2/\arcrange}
      \pgfmathsetmacro\col{\fracprog*100}
      \draw[color={#6!\col!#7},line width=#1] (\i-\progsign:#4+\radrange*\fracprog)
  arc[start angle=\i-\progsign, end angle=\i+1.1*\progsign,radius=#4+\fracprog*\radrange];
    }
  }

  \newcommand{\preassim}[5]{%width, startang, stopang, rad, col
    \pgfmathsetmacro\halfarc{#3/2-#2/2}
    \draw[color={autocatacyc},line width=#1] (#3:#4)
        arc[start angle=#3, end angle=#3-\halfarc,radius=#4];
    \pgfmathsetmacro\startshade{#2}
    \pgfmathsetmacro\endshade{#3-\halfarc}
    \colorgradarc{#1}{\startshade}{\endshade}{#4}{#5}{autocatacyc}
  }

  \newcommand{\postassim}[6]{%width,startang,stopang,rad,col,ratio
    \pgfmathsetmacro\halfarc{#3/2-#2/2}
    \pgfmathsetmacro\progsign{\halfarc>0 ? 1 : -1}
    \pgfmathsetmacro\quarterarc{#3/4-#2/4}
    \pgfmathsetmacro\endshade{#3-\halfarc}
    \pgfmathsetmacro\startbranch{#3-\quarterarc}
    \colorgradarc{#1*#6}{#2}{\endshade}{#4+#1*#6/2-#1/2}{autocatacyc}{#5}
    \draw[color={#5},line width=#1*#6] (\endshade:#4+#1*#6/2-#1/2)
        arc[start angle=\endshade, end angle=\startbranch+\progsign,radius=#4+#1*#6/2-#1/2];
    \draw[color={#5},line width=#1] (\startbranch:#4)
        arc[start angle=\startbranch, end angle=#3,radius=#4];

    \pgfmathsetmacro\arcrange{-\quarterarc}
    \pgfmathsetmacro\outstart{\startbranch+180}
    \pgfmathsetmacro\progsign{\arcrange>0 ? 1 : -1}
    \pgfmathsetmacro\arcend{\outstart+\arcrange-\progsign}
    \pgfmathsetlength\arrowwidth{#1*#6-#1}
    \begin{scope}[shift={(\startbranch:2*#4+#1*#6/2)}]
      \draw[color={#5},line width=#1*#6-#1] (\outstart:#4)
          arc[start angle=\outstart, end angle=\arcend,radius=#4];
      \revarrowhead{\arrowwidth}{\arcend}{#4}{#5}
    \end{scope}
  }

  \newcommand{\assim}[5]{%width,startang,deltaang,rad,ratio
    \pgfmathsetmacro\assimstart{#2+180}
    \begin{scope}[shift={(#2:2*#4+#1*#5/2)}]
      \colorgradarc{#1*#5-#1}{\assimstart}{\assimstart+#3}{#4}{autocatacyc}{assimcol}
    \end{scope}
  }

  \newcommand{\arrowhead}[4]{%width,startang,rad,col
  \fill[#4] (#2+1:#3-#1/2) arc (#2+1:#2:#3-#1/2)
       -- (#2-\protrude:#3) -- (#2:#3+#1/2) arc (#2:#2+1:#3+#1/2) -- cycle;
     }
  \newcommand{\revarrowhead}[4]{%width,startang,rad,col
  \fill[#4] (#2-1:#3-#1/2) arc (#2-1:#2:#3-#1/2)
       -- (#2+\protrude:#3) -- (#2:#3+#1/2) arc (#2:#2-1:#3+#1/2) -- cycle;
     }



\renewcommand{\footnoterule}{}
\tikzset{
  invisible/.style={opacity=0},
  visible on/.style={alt={#1{}{ invisible}}},
  alt/.code args={<#1>#2#3}{%
    \alt<#1>{\pgfkeysalso{#2}}{\pgfkeysalso{#3}} 
  },
}


\usepackage{adjustbox}
\setbeamertemplate{footline}[frame number]

\newcommand{\backupbegin}{
  \newcounter{finalframe}
    \setcounter{finalframe}{\value{framenumber}}
}

\newcommand{\backupend}{
  \setcounter{framenumber}{\value{finalframe}}
}

\title{Investigating physical constraints underlying catalysis and their impact on metabolic systems}
\author{Uri Barenholz}
\institute{CRI Research Symposium}
\date{October 11, 2017}
\usepackage[absolute,overlay]{textpos}
\newcommand\urltext{
    \begin{textblock*}{\paperwidth}(0pt,\textheight)
        \raggedright \small \url{https://git.io/vd2xO} \hspace{.5em}
    \end{textblock*}
}
\begin{document}

\frame{
  \titlepage
    \urltext
}

\frame{\frametitle{Once upon a time\ldots}
    \begin{adjustbox}{max totalsize={0.6\textwidth}{\textheight},center}
        \pgfimage{car.png}
    \end{adjustbox}
    \urltext
}

\frame{\frametitle{Research questions}
\begin{itemize}
    \item What is the physical limit for lowering the activation energy barrier of a given reaction
      \pause
    \item How is the affinity of an enzyme affected by the requirement to be selective
  \end{itemize}
    \urltext
}

\frame{\frametitle{Textbook illustration}
    \begin{adjustbox}{max totalsize={0.6\textwidth}{\textheight},center}
        \pgfimage{energyLandscapeLeninger.png}
    \end{adjustbox}
    \urltext
}

\frame{\frametitle{Modeling energy landscape modification in a classical system}
\only<1>{
    \begin{adjustbox}{max totalsize={0.6\textwidth}{\textheight},center}
      \pgfimage{subsbody2.pdf}
    \end{adjustbox}
  }
\only<2>{
    \begin{adjustbox}{max totalsize={0.6\textwidth}{\textheight},center}
      \pgfimage{subsbody3.pdf}
    \end{adjustbox}
  }
 \only<3>{
    \begin{adjustbox}{max totalsize={0.6\textwidth}{\textheight},center}
      \pgfimage{subsbody4.pdf}
    \end{adjustbox}
  }
 \only<4>{
    \begin{adjustbox}{max totalsize={0.6\textwidth}{\textheight},center}
      \pgfimage{subsbody5.pdf}
    \end{adjustbox}
  }
    \urltext
}

\frame{\frametitle{Reaction energy landscape of model substrate}
\begin{columns}
    \begin{column}{0.5\textwidth}
        \pgfimage[width=\textwidth]{subsbody5.pdf}
    \end{column}
    \begin{column}{0.5\textwidth}
        \pgfimage[width=\textwidth]{energylandscapemodificationmodel.pdf}
    \end{column}
\end{columns}
    \urltext
}

\frame{\frametitle{Introducing a model catalyst}
    \begin{adjustbox}{max totalsize={0.6\textwidth}{\textheight},center}
      \only<1>{
        \pgfimage[width=\textwidth]{subscatdist.pdf}
      }
      \only<2>{
        \pgfimage[width=\textwidth]{subscatdistforce.pdf}
      }
     \end{adjustbox}
    \urltext
}

\frame{\frametitle{Reaction energy landscape of bound substrate}
\begin{columns}
    \begin{column}{0.5\textwidth}
        \pgfimage[width=\textwidth]{subscatforces.pdf}
     \end{column}
    \only<2>{
    \begin{column}{0.5\textwidth}
        \pgfimage[width=\textwidth]{energylandscapemodificationmodel2.pdf}
      \end{column}
    }
\end{columns}
    \urltext
}

\frame{\frametitle{The catalyst creates a bypass to the energy barrier at the transition state}
    \begin{adjustbox}{max totalsize={0.6\textwidth}{\textheight},center}
        \pgfimage{catlandscape.pdf}
    \end{adjustbox}
    \urltext
}

\frame{\frametitle{Subtracting the potential field at the transition state from the initial state produces an energy barrier reduction landscape}
    \only<1-2>{
\begin{columns}
    \begin{column}{0.5\textwidth}
      \pgfimage[width=\textwidth]{subsstartpot.pdf}
    \end{column}
    \begin{column}{0.5\textwidth}
    \only<2>{
        \pgfimage[width=\textwidth]{substranspot.pdf}
    }
    \end{column}
\end{columns}
}
    \only<3>{
    \begin{adjustbox}{max totalsize={0.6\textwidth}{\textheight},center}
        \pgfimage{subsdiffpot.pdf}
    \end{adjustbox}
     }
    \urltext
}

\frame{\frametitle{Subtracting the potential field at the transition state from the initial state produces an energy barrier reduction landscape}
\begin{itemize}
    \item The resulting function quantifies the barrier reduction when positioning a positive point charge at any coordinate in space
      \pause
    \item Placing charges at extremum points of this function achieves maximal barrier reduction
  \end{itemize}
    \urltext
}


\frame{\frametitle{Methodological approach for investigating catalytic constraints}
\begin{itemize}
      \pause
     \item Crowd-sourcing platform
      \begin{itemize}
           \item Challenge existing assumptions
           \item Reveal potential catalytic mechanisms
      \end{itemize}
      \pause
     \item Handheld, functional models
      \begin{itemize}
       \item Demonstrate and communicate catalytic principles
      \end{itemize}
      \pause
     \item Apply theoretical framework to molecular domain
      \pause
     \item Investigate metabolic network design implications
      \begin{itemize}
       \item Synthetic biology applications
       \item Origins of life metabolism
      \end{itemize}
  \end{itemize}
    \urltext
}

\frame{\frametitle{Does structural similarity limit affinity in metabolic networks?}
\pause
\vfill
\vfill
    \begin{adjustbox}{max totalsize={0.6\textwidth}{\textheight},center}
        \pgfimage{babyToy.jpg}
    \end{adjustbox}
    \urltext
}

\frame{\frametitle{Does structural similarity limit affinity in metabolic networks?}
\begin{itemize}
  \item Most enzymes are substrate-specific
    \pause
   \item Structural similarity is used for drug discovery and promiscuous activity tests
     \pause
   \item Metabolic networks must contain structurally similar metabolites
    \begin{itemize}
         \item But can potentially reduce similarities at critical points
     \end{itemize}
     \pause
     \item Numerous examples for specificity tradeoffs in the literature
  \end{itemize}
    \urltext
}

\frame{\frametitle{Why do we expect selectivity to decrease affinity?}
    \only<1>{
    \begin{adjustbox}{max totalsize={0.6\textwidth}{\textheight},center}
        \pgfimage{twoshapes.pdf}
      \end{adjustbox}}
      \only<2>{
    \begin{adjustbox}{max totalsize={0.6\textwidth}{\textheight},center}
        \pgfimage{twoshapesslack.pdf}
      \end{adjustbox}}
     \urltext
}


\frame{\frametitle{Examples of specificity-affinity challenges}
\begin{columns}
    \begin{column}{0.5\textwidth}
 \begin{itemize}
    \only<1->{
     \item RuBisCo
      \begin{itemize}
        \item \ce{CO2} versus \ce{O2}
      \end{itemize}
    }
    \only<2->{
     \item Tyrosine ammonia lyase
      \begin{itemize}
        \item \ce{Tyr} versus \ce{Phe}
      \end{itemize}
    }
    \only<3->{
    \item Bacterial DNA methyltransferase
      \begin{itemize}
        \item Relaxing sequence specificity accelerates rate
      \end{itemize}
    \item Bacterial hexose phosphate transporter
    }
       
  \end{itemize}
    \end{column}
    \begin{column}{0.5\textwidth}
    \only<1>{
    \begin{adjustbox}{max totalsize={0.6\textwidth}{\textheight},center}
        \pgfimage{co2o2.jpg}
      \end{adjustbox}}
    \only<2>{
    \begin{adjustbox}{max totalsize={0.6\textwidth}{\textheight},center}
        \pgfimage{tyrosinePhenylalenine.jpg}
      \end{adjustbox}}
      \end{column}
\end{columns}
    \urltext
}

\frame{\frametitle{Can we formulate a quantitative evaluation of the selectivity challenge?}
\begin{itemize}
     \item Given metabolites concentration data
      \begin{itemize}
       \item Identify challenging reactions
       \item Quantify expected cost
      \end{itemize}
      \pause
     \item Given reaction possibilities
      \begin{itemize}
       \item Find biases in metabolic network structure maximizing structural differences
      \end{itemize}
  \end{itemize}
    \urltext
}

\frame{\frametitle{Methodological approach for investigating selectivity tradeoffs}
\begin{itemize}
     \item Impact on metabolites concentrations and enzymes
   \begin{itemize}
       \item BRENDA - identifying weak affinity enzymes
       \item Promiscuous activity data from Sauer lab
       \item Structural similarity metrics comparison with measured metabolites concentrations
   \end{itemize}
   \pause
 \item Impact on network structure
   \begin{itemize}
     \item Project metabolic networks to chemical space
     \item Implement selectivity in constraint based modeling of metabolic networks
   \end{itemize}
 \end{itemize}
    \urltext
}
 
\frame{\frametitle{Summary}
\begin{itemize}
  \item Basic challenges of biological systems are rarely investigated theoretically
     \item Transforming key problems to simplified models in accessible platforms can leverage innovation of wider audience and reveal novel principles
     \item Recently available datasets allow evaluation of hypotheses
     \item Mapping metabolic networks into the chemical space can highlight metabolic network motifs

\end{itemize}
Thank You!
  \urltext
}
 
\backupbegin

\frame{\frametitle{Work plan}
    \begin{adjustbox}{max totalsize={\textwidth}{\textheight},center}
        \pgfimage{gantt.pdf}
    \end{adjustbox}
    \urltext
}

\frame{\frametitle{Catalyst design must track the entire reaction pathway}
    \begin{adjustbox}{max totalsize={0.6\textwidth}{\textheight},center}
        \pgfimage{energylandscapemodificationmodelMax.pdf}
    \end{adjustbox}
    \urltext
}

\frame{\frametitle{References}
\scriptsize{
\begin{enumerate}
  \item Cooper S, et al. (2010) Predicting protein structures with a multiplayer online game. Nature
  \item Cao Y, et al. (2008) ChemmineR: a compound mining framework for R. Bioinformatics 
  \item Reymond J-L (2015) The chemical space project. Acc Chem Res
  \item Wang Y, et al. (2013) fmcsR: mismatch tolerant maximum common substructure searching in R. Bioinformatics
  \item Bar-Even A, et al. (2015) The Moderately Efficient Enzyme: Futile Encounters and Enzyme Floppiness. Biochemistry
  \item Alam MT, et al. (2017) The self-inhibitory nature of metabolic networks and its alleviation through compartmentalization. Nat Commun
  \item Schomburg I, et al. (2004) BRENDA, the enzyme database: updates and major new developments. Nucleic Acids Res
  \item Sévin DC, et al. (2017) Nontargeted in vitro metabolomics for high-throughput identification of novel enzymes in Escherichia coli. Nat Methods
  \item Savir Y, et al. (2010) Cross-species analysis traces adaptation of Rubisco toward optimality in a low-dimensional landscape. PNAS
  \item Tcherkez GGB, et al. (2006) Despite slow catalysis and confused substrate specificity, all ribulose bisphosphate carboxylases may be nearly perfectly optimized. PNAS
  \item Danos V, et al. (2015) Rigid Geometric Constraints for Kappa Models. Electron Notes Theor Comput Sci
\end{enumerate}}
    \urltext
}

\backupend

\end{document}
